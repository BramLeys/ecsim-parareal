\documentclass[master=wit, english]{kulemt}
\setup{% Remove the "%" on the next line when using UTF-8 character encoding
  inputenc=utf8,
  title={Parallelisation in time of particle-in-cell plasma simulations using the parareal algorithm},
  author={Bram Leys},
  promotor={Prof.\,Dr.\,G. Lapenta\and Prof.\,Dr.\,Ir.\,D. Huybrechs},
  assessor={Prof.\,Dr.\,Ir.\,R. Sepulchre\and Prof.\,Dr.\,Ir.\,S. Vandewalle },
  assistant={Dr.\,P.J. Deka}}
% Remove the "%" on the next line for generating the cover page
%\setup{coverpageonly}
% Remove the "%" before the next "\setup" to generate only the first pages
% (e.g., if you are a Word user).
%\setup{frontpagesonly}

% Choose the main text font (e.g., Latin Modern)
\setup{font=lm}

% If you want to include other LaTeX packages, do it here. 
\usepackage{graphicx}
\usepackage[thinc]{esdiff}
\usepackage{amsmath}
\usepackage{amsfonts}
\usepackage{float}
\usepackage{subcaption}
% \usepackage[round, authoryear, sort]{natbib}

% Finally the hyperref package is used for pdf files.
% This can be commented out for printed versions.
\usepackage[pdfusetitle,colorlinks,plainpages=false, linkcolor=blue, citecolor=blue, urlcolor=blue]{hyperref}

%%%%%%%
% The lipsum package is used to generate random text.
% You never need this in a real master's thesis text!
\IfFileExists{lipsum.sty}%
 {\usepackage{lipsum}}%
 {\newcommand{\lipsum}[1][1-7]{\par And some text: lipsum ##1.\par}}
%%%%%%%

%\includeonly{chap-n}
\begin{document}

\begin{preface}
 First and foremost, I would like to express my gratitude to my supervisor, Prof.\,Dr.\, Giovanni Lapenta, who, tragically, passed away during the making of this thesis after a valiant battle against cancer. Despite the harsh effects of his illness, he continued to guide and counsel me to the best of his ability and for this, I will be forever grateful.
 
I also extend my heartfelt appreciation to my assistant supervisor, Dr.\,Pranab J Deka. His unwavering support and insightful guidance have been instrumental in shaping this thesis. His contributions have not only improved this research but have also had a profound impact on my development as a researcher.
 
 Next, I wish to thank my co-supervisor, Prof.\,Dr.\,Ir.\,Daan Huybrechs, who immediately stepped in to support me following the passing of Prof. Lapenta. His support and stability during this challenging time have been invaluable. 
 
I am also thankful to my assessors, Prof.\,Dr.\,Ir.\,Rodolphe Sepulchre and Prof.\,Dr.\,Ir.\, Stefan Vandewalle, for their critical feedback and valuable insights. Their input has significantly enhanced the quality of this work.
 
 Lastly, I would like to thank my parents, family and friends for their unwavering support and patience not only during my studies, but always. Their belief in me provided the strength and motivation needed to complete this academic journey.
 
\end{preface}

\tableofcontents*

\begin{abstract}
This thesis explores the efficient implementation of plasma simulation techniques for high-performance computing systems. We investigate time parallelism for particle-in-cell simulations using the parareal algorithm with the energy conserving semi-implicit method. In this work, we consider shared-memory parallelism on a single compute node with \texttt{OpenMP}. We examine various coarse and fine solvers within the parareal framework to identify optimal combinations.

Our research begins with investigating temporal coarsening, which shows promising results for highly accurate solutions. Then, we explore subcycling within the coarse solver, although our findings indicate that overhead often negates any speedup. We discuss the trade-offs associated with an increase in the number of computational cores used and diminishing parallel efficiency. Finally, we evaluate different combinations of linear solvers and preconditioners in the energy conserving semi-implicit method used for the coarse and fine integrator.

Our conclusions suggest that significant speedup is achievable when applying the parareal algorithm to hyperbolic problems, provided that appropriate coarse and fine solvers are selected. Notably, the coarse solver must be sufficiently accurate to avoid issues related to accuracy inhibiting the convergence of parareal. 
\end{abstract}

\begin{abstract*}
Deze thesis onderzoekt de efficiënte implementatie van plasmasimulatietechnieken voor high-performance computing systemen. We focussen op tijdparallellisme voor particle-in-cell simulaties. Dit wordt specifiek gedaan met behulp van het parareal algoritme dat we toepassen op de energy conserving semi-implicit method. In dit werk beschouwen we shared-memory parallellisme op één node met behulp van \texttt{OpenMP}. We onderzoeken verschillende grove en fijne tijdstap methoden binnen de context van parareal om optimale combinaties te kunnen identificeren.

Ons onderzoek begint met het onderzoeken van tijdsstap vergroting in de grove tijdstap methode, dit laat veelbelovende resultaten zien voor simulaties met hoge precisie. Vervolgens onderzoeken we het toepassen van subcycling tijdens de grove tijdstap methode, onze bevindingen geven echter aan dat overhead vaak enige mogelijke tijdsverbetering teniet doet. We bespreken de afwegingen omtrent de toename van het aantal gebruikte processoren en de afnemende parallelle efficiëntie. Tot slot evalueren we verschillende combinaties van methoden om lineaire systemen op te lossen in de fijne en grove tijdstap methoden. Deze methoden worden gebruikt in combinatie met mogelijke preconditioners in de energy conserving semi-implicit method.

Onze conclusies suggereren dat een aanzienlijke versnelling haalbaar is bij het toepassen van het parareal algoritme op hyperbolische problemen, indien de juiste grove en fijne integratoren worden gekozen. De grove integrator moet voldoende nauwkeurig zijn om te voorkomen dat nauwkeurigheidsproblemen de convergentie van parareal belemmeren. 
\end{abstract*}

% A list of figures and tables is optional
%\listoffigures
%\listoftables
% If you only have a few figures and tables you can use the following instead
\listoffiguresandtables
% The list of symbols is also optional.
% This list must be created manually, e.g., as follows:
\chapter{List of Abbreviations and Symbols}
\section*{Abbreviations}
\begin{flushleft}
  \renewcommand{\arraystretch}{1.1}
  \begin{tabularx}{\textwidth}{@{}p{12mm}X@{}}
  	PIC & Particle-in-Cell\\
    ECSIM   & Energy Conserving Semi-Implicit Method \cite{lapenta_exactly_2017}\\
    ODE & Ordinary Differential Equation\\
    PDE & Partial Differential Equation\\
    CFL  & Courant-Friedrichs-Lewy \cite{courant_uber_1928} \\
    % PFASST   & Parallel Full Approximation Scheme in Space and Time \cite{emmett_toward_2012}\\
    % SDC & Spectral Deferred Corrections \cite{dutt_spectral_2000}\\
    % FAS & Full Approximation Scheme \cite{brandt_multi-level_1976} \\
    CN & Crank--Nicolson \cite{Crank_Nicolson_1947} 
    % GMRES & Generalised Minimal Residiual \cite{youcef_saad_martin_h_schultz_gmres_1986}\\
    % BiCGSTAB & Biconjugate Gradient Stabilised \cite{van_der_vorst_bi-cgstab_1992}\\
  \end{tabularx}
\end{flushleft}
\section*{Symbols}
\begin{flushleft}
  \renewcommand{\arraystretch}{1.1}
  \begin{tabularx}{\textwidth}{@{}p{12mm}X@{}}
    $\textbf{E}$    & Electric field \\
    $\textbf{B}$   & Magnetic field \\
    $\textbf{J}$ & Current\\
    $\textbf{v}$   & Velocity \\
    $\textbf{x}$   & Position \\
    $\textbf{F}$ & Fine solver \\
    $\textbf{G}$ & Coarse solver \\
    $q$ & Electric charge\\
    $m$ & Mass\\
    $\rho$ & Charge density\\
    $c$ & The speed of light\\
    $\Delta t$ & Time step size\\
    $\Delta x$ & Grid cell size\\

    
  \end{tabularx}
\end{flushleft}

\newcommand{\pjd}[1]{{{\color{blue} #1}}}
\newcommand{\bram}[1]{{{\color{red} #1}}}

% Now comes the main text
\mainmatter

\chapter{Introduction}
\label{cha: intro}

\pjd{enumerate equations as needed}


This first chapter describes the background of the thesis as well as the goals and modus operandi. In section \ref{sec: intro plasma}, plasma is described, how it is formed and some of its characteristics. Different approaches to calculating the evolution of plasma are also briefly described. 
These plasma dynamics can be described using differential equations, section \ref{sec: DE} therefore briefly gives some background on these equations and how they are traditionally solved. 
This chapter ends with section \ref{sec: goals and approach} where the goals and methodology of the rest of the paper are presented.

The rest of the thesis is subdivided into four other chapters. 
In chapter \ref{cha: pic} special attention is given to particle-in-cell methods since these are the type of plasma simulation algorithm used in this work.  
Parallel-in-time methods are introduced in chapter \ref{cha: pint}, which will be used to speed up the calculation of the chosen particle-in-cell method. 
Chapter \ref{cha: methods and results} reports on the performed experiments. These will demonstrate desired properties such as speedup, accuracy and conservation of energy. 
The final chapter \ref{cha:conclusion} contains the conclusions of this thesis.


\section{Plasma}
\label{sec: intro plasma}
\color{red}
Plasma is a state of matter separate from solid, fluid and gas, where the atoms or molecules have been excited to high enough energies that the electrons are being stripped away from the nuclei. In our daily lives, we might encounter plasma a couple of times, for example, in lightning or neon signs, but in space, it is the most common state for matter, making up about 99.9\% of the observable universe\cite{noauthor_plasma_nodate}. The sun, for example, is made out of plasma, and current research is trying to replicate the fusion that occurs in the sun's plasma to satisfy our energy demands \cite{degrave_magnetic_2022}.  
\newline
Plasma can be characterized in a couple of ways, one of which is its Debye length, $\lambda_D$. This value describes how the electric potential changes around a point charge:
\[\varphi(x) = \varphi_0 e^{-\frac{x}{\lambda_D}}\]  
This indicates that plasma creates a shield around any introduced local charges, as to minimise influence at large distances from this perturbation. If $n$ defines the density of the plasma, then the number of particles in a cube, whose sides are as long as the Debye length, is equal to $N_{D} = n \lambda_D^3$. This number, in turn, is used to calculate the plasma parameter $\Lambda = 6 \pi N_D^{\frac{2}{3}}$, which is a measure of the plasma coupling. If $N_D$ is small, there would only be a couple of particles nearby at each point in space. This means that if another particle gets closer or one of the present particles goes further, then the potential energy at that point would sharply increase or decrease, respectively. In this case, the electrostatic potential is dominant, and the trajectories of the species are \textbf{strongly} influenced by the interaction with close particles.
 A plasma with many particles in its Debye length is considered weakly coupled. In this case, the movements of many particles determine the electrostatic potential at each point in space. This means that whether one particle leaves or enters the area of influence does not impact the overall field much. Only the behaviour of the plasma at large matters since the individual particles only \textbf{weakly} impact each other. As a result, the trajectories of the species are, in fact, determined mostly by the kinetic energy of the particles.\cite{giovanni_lapenta_introduction_nodate}
\newline
A final characteristic is the plasma frequency, $\omega_{pe}$. It describes the frequency of plasma oscillations or Langmuir waves created when a cloud of electrons is displaced from its equilibrium position. Neglecting the influence of temperature and assuming immobile ions, it can be written as
 \[\omega_{pe} = \left(\frac{n_e q_e^2}{m_e\epsilon_0}\right)^{\frac{1}{2}}\]
 where $n_e$ is the density of electrons, $q_e$ is the charge and $m_e$ is the mass of an electron. The Langmuir waves are among the waves with the highest frequency and indicate how fast a plasma will react to disturbances.\cite{giovanni_lapenta_introduction_nodate}

\subsection{Time evolution approaches}
\label{subsec: plasma approaches}
Three different ways of calculating the state and evolution of plasma are described: the kinetic, fluid and particle-in-cell approaches. The kinetic description is very powerful at describing the evolution at the microscopic level. It gives a statistical view of the distribution of positions and velocities of the particles in the plasma. The Boltzmann equation states how this distribution can change in time due to external forces, diffusion and internal collisions:
\[\diff{f(\textbf{x}(t), \textbf{v}(t),t))}{t} =\left(\diff{f}{t}\right)_{\mathrm{force}}+\left(\diff{f}{t}\right)_{\mathrm{diff}}+ \left(\diff{f}{t}\right)_{\mathrm{coll}}\]
Where $\textbf{v}(t)$ and $\textbf{x}(t)$ represent the velocity and position at time $t$, respectively.
$f(\textbf{x}(t), \textbf{v}(t),t))$ is the statistical distribution function of the plasma and has the property that $\int f(\textbf{x}(t),\textbf{v}(t),t)d\textbf{v} = n(\textbf{x}(t),t)$ where $n(\textbf{x}(t),t)$ is the density of the plasma in a box with sides of length $d\textbf{x}$ around $\textbf{x}(t)$. Using Liouville's theorem, however, it can be reduced to:\[\diff{f(\textbf{x}(t), \textbf{v}(t),t))}{t} = \left(\diff{f}{t}\right)_{\mathrm{coll}}\]
Using Newton's laws of motion and expanding the derivative with forces from both gravity and the electromagnetic field gives:
\[\diffp{f}{t} + \textbf{v} \diffp{f}{{\textbf{x}}} + \left( \frac{\textbf{F}}{m} + \frac{q}{m}(\textbf{E} + \textbf{v} \times \textbf{B})\right)\diffp{f}{{\textbf{v}}}= \left(\diff{f}{t}\right)_{\mathrm{coll}}\]
 $q$ represents the charge and $m$ the mass of the species in question.
While this notation leads to accurate simulations at small scales, it can be prohibitively computationally expensive. Especially finding the $\left(\diff{f}{t}\right)_{\mathrm{coll}}$ term can require extensive knowledge about the problem. The latter problem is less of an issue in very weakly coupled systems since, in this case, the collision term can be set to 0, leading to the Vlasov equation.
\newline The fluid approach seeks instead to describe the macroscopic behaviour of the plasma in terms of wave dynamics. The general form of the wave equation can be written as \[\psi(x,t) = \tilde{\psi}e^{-i\omega t + i k x}\] where $\psi$ is the desired quantity, k the wave number and $\omega$ the angular velocity, need to be found.
\newline
Although this representation is less computationally expensive, certain electrostatic phenomena can only be simulated by considering the evolution at a small scale \cite{biskamp_magnetic_2000}. Therefore, a simulation strategy must be found to simulate these large-scale phenomena while not explicitly resolving the small time scales. This can be done using the (semi-)implicit particle-in-cell, PIC, family of methods. \cite{giovanni_lapenta_introduction_nodate}
\color{black}


\section{Differential Equations}
\label{sec: DE}
The equations that describe the dynamics of plasma are called differential equations. These are equations that describe the dynamics of a certain metric of interest compared to the evolution of another metric, often time or space. When the variable is only dependent on one metric, it is called an ordinary differential equation, ODE. A generalised ODE in function of $t$ can be written as 
\begin{equation}
\label{eq: ODE}
    \frac{d \textbf{U}(t)}{d t} = \textbf{f}(\textbf{U}(t),t) \quad \text{with} \quad \textbf{U}(0) = \textbf{U}_0
    \end{equation}
where $\textbf{U}(t) \in \mathbb{R}^n$ is an n-dimensional state vector and $\textbf{f}(\textbf{U}(t), t)$ is an m-dimensional system of equations. Systems with higher-order derivatives can be transformed into a larger system with only first-order derivatives. For example the second-order differential equation
\begin{equation*}
    \frac{d^2 \textbf{x}(t)}{d t^2} = \textbf{a}(\textbf{x}(t),t) \quad \text{with} \quad \textbf{x}(0) = \textbf{x}_0, \quad \frac{d \textbf{x}(0)}{d t} = \textbf{v}_0
\end{equation*}
can be transformed into
\begin{equation*}
    \left\{ \begin{aligned} 
      \frac{d \textbf{x}(t)}{d t} &= \textbf{v}(t) \\
      \frac{d \textbf{v}(t)}{d t} &= \textbf{a}(\textbf{x}(t),t)
    \end{aligned} \right.
    \quad \text{with} \quad \textbf{x}(0) = \textbf{x}_0, \quad \textbf{v}(0) = \textbf{v}_0
\end{equation*}
where $\textbf{U}(t) = (\textbf{x}(t), \textbf{v}(t))$.

In an even more general case, the variable of interest can depend on multiple parameters and only partial derivatives are given to each. An example is the diffusion equation
\begin{equation}
\label{eq: general diffusion}
    \frac{\partial \textbf{U}(\textbf{x},t)}{\partial t} = \nabla \cdot (a(\textbf{U}(\textbf{x},t),\textbf{x}) \nabla \textbf{U}(\textbf{x},t))
    \end{equation}
\color{red} Explain difference hyperbolic and parabolic systems \color{black}
Since these equations occur frequently when describing systems, much research has been performed on them. While some can be solved analytically, the larger part must be solved using numerical methods. 
Assume an ODE, depending on time, needs to be simulated from [0, T]. These methods traditionally divide the requested parameter range into $N + 1$ chunks, $0 = t_0 < t_1 < t_2 < \hdots < t_{N} = T$. It is then assumed that the functions $\textbf{f}(\textbf{U}(t),t)$ remain constant over these smaller parameter ranges. The smaller these steps are, the more they should approximate the continuous solution. The relation between finer steps and a more accurate solution is rarely one-to-one, but can often be approximated by a convergence ratio. For example, a convergence of $\mathcal{O}(\Delta t^2)$ indicates that one expects that if the distance between $t_i$ and $t_{i+1}$, $\Delta t$, is halved then the error compared to the actual solution is multiplied by $\frac{1}{2}^2$. The exponent in the convergence bound is used to classify the method, e.g. second-order accuracy. 
In general, the solution at a time step $n$ can be written as
\[
\textbf{U}_n = \sum_{i=0}^N a_i g(\textbf{U}_i,t_i)
\]
Two main distinctions can be made based on which time points are needed for a method. Explicit methods only require previous time steps, meaning $i < n$. Implicit methods require solutions at later points in time. They both have their respective use cases. Explicit methods are, in general, faster than their implicit counterparts, but some can become unstable when large timesteps are used. Implicit methods on the other hand need to perform an iteration scheme to obtain their solution, but their solution will remain stable for any chosen timestep.
 


\section{Goals and Approach}
\label{sec: goals and approach}
The main goal of this thesis is an efficient implementation of a PIC method on a massively parallel computer system. Plasma simulations calculated using this method should have good energy-conserving properties. First, a PIC method will be selected, which incorporates the energy-conserving property. 
Since much research has already been done on the parallelization in space for PIC methods, this work will focus on time parallelism. Different parallel-in-time methods will be investigated to choose a suitable algorithm. The PIC and PinT methods will then be combined and investigated to improve performance and energy-conserving properties.


%%% Local Variables: 
%%% mode: latex
%%% TeX-master: "thesis"
%%% End: 

\chapter{Particle-in-Cell Simulations of Plasma}
\label{cha: pic}
This chapter describes the ideas behind the family of plasma simulation methods used in this thesis: PIC methods. After a general introduction, we describe the specific choice of method: the energy-conserving semi-implicit method by Lapenta (2017) \cite{lapenta_exactly_2017}.

\section{General Outline}
PIC methods are closely related to the kinetic approach. During PIC the evolution of the statistical distribution function is simulated using the dynamics of a sampled distribution. PIC methods use the Vlasov equation to simulate the plasma dynamics and thus assume collisionless plasma. This assumption is warranted for weakly coupled plasmas. To simulate such systems, it would be computationally infeasible to follow each particle. As such PIC combines many particles close to each other in phase space into superparticles. These superparticles, representing many physical particles, have a finite shape, and their interaction with others weakens as their overlap increases. This behaviour allows the simulated system to retain the properties of the weakly coupled system it represents. The shape of a superparticle, $p$, is determined in phase space by \textbf{shape functions} for \textbf{x} and \textbf{v}, $S_{\textbf{x}}(\textbf{x}-\textbf{x}_p(t))$ and $S_{\textbf{v}}(\textbf{v}-\textbf{v}_p(t))$ respectively. It is assumed that the actual physical distribution can be represented as the superposition of these superparticles \cite{markidis_multi-scale_2010}
\begin{equation}
    f(\textbf{x}(t),\textbf{v}(t),t) = \sum_{p}N_pS_{\textbf{x}}(\textbf{x}-\textbf{x}_p(t))S_{\textbf{v}}(\textbf{v}-\textbf{v}_p(t))
\end{equation}
The dynamics of these particles follow the laws of motion
\begin{equation} \label{eq: cont particle mover}
\left\{\begin{aligned} 
	\diff{{\textbf{x}_p}}{t} &= \textbf{v}_p \\
	\diff{{\textbf{v}_p}}{t} &= \frac{q_p}{m_p}\left(\textbf{E} + \frac{\textbf{v}_p\times \textbf{B}}{c}\right)
\end{aligned}\right.
\end{equation}
The influence of the electric field, $\textbf{E}$, and magnetic field $\textbf{B}$ are due to the Lorentz force. 
These equations describe the movement of the particles in the spatial domain in relation to the time domain, and as such the calculation of the positions and velocities of the particles is referred to as the \textbf{particle mover} part. The movement and position of the particles also define a current density, \textbf{J}, and charge density, $\rho$, at each point in time and space. These values, in turn, influence the magnetic and electric fields. The section corresponding to the computation of the electric and magnetic fields is called the \textbf{field solver}. The electric and magnetic fields are calculated by simulating the Maxwell equations (\ref{eq: gauss}, \ref{eq: gauss magnetic}, \ref{eq: Faraday}, \ref{eq: Ampere}). 

PIC uses a semi-Lagrangian/semi-Eulerian discretisation scheme. As in a Lagrangian scheme, the particles are tracked during the simulation; however, they are coupled to the electromagnetic fields, which are calculated on an Eulerian grid. 

\section{Numerical Implementations}
To simulate the Maxwell equations in practice we perform a discretisation in both space and time. Typically, the domain is discretised in space using a regular grid. On this grid, the electric field is often stored at the vertices, while the magnetic field is kept at the centres. This creates a "staggered" grid in the spatial domain. Particles can then move across this grid, and the force experienced by these particles is computed by interpolating the electric and magnetic fields at the neighbouring grid points. The standard \textbf{interpolation function} of a particle, $p$, to a grid point, $g$, is defined as 
\begin{equation}
    W(\textbf{x}_p - \textbf{x}_g) = \int S_{\textbf{x}}(\textbf{x}-\textbf{x}_p)b_0\left(\frac{\textbf{x}-\textbf{x}_g}{\Delta\textbf{x}}\right) d\textbf{x}
\end{equation}
where $b_0$ is the b-spline of order 0. If the particle shape function is a b-spline of order $l$ and the same size as a grid cell, $S_\textbf{x} = \frac{1}{\Delta x}b_l\left(\frac{\textbf{x}-\textbf{x}_p}{\Delta x}\right)$, the interpolation function can be simplified using the properties of b-splines.
\begin{equation}
  W(\textbf{x}_p - \textbf{x}_g) = W_{pg} = b_{l+1}\left(\frac{\textbf{x}_p-\textbf{x}_g}{\Delta \textbf{x}}\right)  
\end{equation}
The shape functions are often b-splines of order 0, making the interpolation function a b-spline of order 1. 
The different numerical integration schemes of the differential equations set apart the different PIC methods. For example, a simple algorithm could use the \textbf{leap-frog scheme} for the particle mover
\begin{equation}
\left\{\begin{aligned} 
	\textbf{x}_p^{n+\frac{1}{2}} &= \textbf{x}_p^{n-\frac{1}{2}} + \Delta t \textbf{v}_p^{n} \\
	\textbf{v}_p^{n+1} &= \textbf{v}_p^{n} + \Delta t \frac{q_p}{m_p}\left(\textbf{E}^{n+\frac{1}{2}}(\textbf{x}_p^{n+\frac{1}{2}}) + \bar{\textbf{v}}_p \times \textbf{B}^{n+\frac{1}{2}}(\textbf{x}_p^{n+\frac{1}{2}})\right)\\
\end{aligned}\right.
\end{equation}
where $\bar{\textbf{v}}_p=\frac{\textbf{v}_p^{n+1}+\textbf{v}_p^{n}}{2}$ and $\textbf{E}^{n+\frac{1}{2}}(\textbf{x}_p^{n+\frac{1}{2}})$ and $\textbf{B}^{n+\frac{1}{2}}(\textbf{x}_p^{n+\frac{1}{2}})$ are the electric and magnetic fields, respectively, at time $t_{n+\frac{1}{2}}$ and position $\textbf{x}_p^{n+\frac{1}{2}}$. These are calculated numerically as the interpolated values from each gridpoint to the position of the particle in question
\begin{equation}
\begin{split}
    \textbf{E}^{n+\frac{1}{2}}(\textbf{x}_p^{n+\frac{1}{2}}) &= \sum_g \textbf{E}_g^{n+\frac{1}{2}} W_{pg} = \textbf{E}_p^{n+\frac{1}{2}}\\
    \textbf{B}^{n+\frac{1}{2}}(\textbf{x}_p^{n+\frac{1}{2}}) &= \sum_g \textbf{B}_g^{n+\frac{1}{2}} W_{pg} = \textbf{B}_p^{n+\frac{1}{2}}
\end{split}
\end{equation} 

Crucially, the position and velocity are calculated at different time steps for the leap-frog scheme. This staggering makes the method second-order accurate in time as it calculates a centred finite difference. The electric and magnetic fields can be computed by solving the equations \cite{jiang_origin_1996}
\begin{equation}
\left\{\begin{aligned}
	\nabla_g \times \mathbf{E}^{n} +\frac{1}{c}\frac{\mathbf{B}^{n+1} - \mathbf{B}^{n}}{\Delta t} &= 0 \\
	\nabla_g \times \mathbf{B}^{n} -\frac{1}{c}\frac{\mathbf{E}^{n+1} - \mathbf{E}^{n}}{\Delta t} &= \frac{4 \pi}{c}\bar{\mathbf{J}}_g\\
\end{aligned}\right.
\end{equation}
Depending on how the current density at each grid point, $\bar{\mathbf{J}}_g$, is calculated also heavily influences the properties of the method. Energy conservation heavily depends on accurately capturing the non-linear interaction between particles and fields. If energy is not conserved, then extra care must be taken so that this method's results would still accurately represent the physical system. While multiple PIC methods are energy-conserving, most are iterative and require some form of linear or non-linear iterations.

We describe one final special technique of PIC methods called subcycling. If subcycling is used, the field solver is not always performed when the particle mover is simulated. Instead, one performs the particle mover multiple times and uses the average value of these time steps when performing the field solver. This cuts down on the computational cost in exchange for accuracy. As particles typically follow a gyrating path in the presence of electromagnetic fields \cite{chen_lecture_2003}, subcycling can also be used to select a large time step size to step over the gyration cycle. The position and velocity can then be computed at stages during the gyration, giving an average over the gyromotion while calculating the fields. This is especially useful when the particle movement is much faster than the field dynamics. Without subcycling, the smallest time scale of the two would need to be chosen for the simulation. This means a very fine time scale might need to be used to simulate slowly evolving fields, which would be unnecessary. With subcycling, the particles can be simulated at their respective time scale, while the fields can be computed with a larger time step size.
\cite{chen_implicit_2023, lapenta_advances_2023}

\section{Numerical Constraints}
As PIC methods involve the discretisation of PDEs, they also often suffer from limitations on the grid sizes for the temporal and spatial domains. As the Langmuir waves are typically the fastest oscillations in a plasma, the time step size must allow the method to resolve these scales. Based on the Nyquist condition this can be translated into the following requirement
\begin{equation}
    \omega_{pe}\Delta t < 2
\end{equation}
In the spatial domain, the electron Debye length puts a constraint on the maximal grid cell size
\begin{equation}
    \Delta x < \zeta \lambda_D
\end{equation}
where $\zeta$ is a constant value of order one defined by the used PIC method. 
This avoids the so-called \textbf{finite grid instability} that arises due to aliasing Fourier modes \cite{giovanni_lapenta_introduction_nodate}. On top of these two requirements, explicit PIC methods can also suffer from a CFL condition which limits the ratio between $\Delta t$ and $\Delta x$ to
\begin{equation}
    \frac{c \Delta t}{\Delta x} < C
\end{equation}
where C is a constant value defined by the used PIC (often 1). 
This constraint can be interpreted by imagining a particle or wave moving at a speed $\textbf{v}$ over the spatial grid. The lefthand side gives the number of grid cells such a particle or wave would cross in a single time step. The righthand side of the constraint defines how many grid cells can be skipped before the method turns unstable. This indicates that there should also be a requirement using the maximal velocity of the particles, however, we assume that the waves travel at the speed of light in a vacuum and that particles move slower than $c$. This assures that $\frac{c \Delta t}{\Delta x} < C$ gives the most stringent condition.

\section{Energy Conserving Semi-Implicit Method}
\label{subsec: plasma intro ECSIM}
The energy-conserving semi-implicit method (ECSIM), developed by Lapenta (2017) \cite{lapenta_exactly_2017}, is fully energy-conserving, has no finite grid instability and only requires a linear solver \cite{lapenta_exactly_2017}. It is based on the iPIC3D method of Markidis et al. (2010) \cite{markidis_multi-scale_2010} and the energy-conserving PIC $\theta$-scheme by Brackbill et al. (1982) \cite{brackbill_implicit_1982}. The particle mover is given by 
\begin{equation}
\left\{\begin{aligned} 
	\textbf{x}_p^{n+\frac{1}{2}} &= \textbf{x}_p^{n-\frac{1}{2}} + \Delta t \textbf{v}_p^{n} \\
	\textbf{v}_p^{n+1} &= \textbf{v}_p^{n} + \Delta t \frac{q_p}{m_p}\left(\textbf{E}^{n+\theta}_p(\textbf{x}_p^{n+\frac{1}{2}}) + \bar{\textbf{v}}_p \times \textbf{B}^n(\textbf{x}_p^{n+\frac{1}{2}})\right)\\
 \end{aligned}\right.
\end{equation}
and its field solver is given by
\begin{equation}
\left\{\begin{aligned} 
 	\nabla_g \times \mathbf{E}^{n + \theta} +\frac{1}{c}\frac{\mathbf{B}_g^{n+1} - \mathbf{B}_g^{n}}{\Delta t} &= 0 \\
 	\nabla_g \times \mathbf{B}^{n+ \theta} -\frac{1}{c}\frac{\mathbf{E}_g^{n+1} - \mathbf{E}_g^{n}}{\Delta t} &= \frac{4 \pi}{c}\bar{\mathbf{J}}_g\\
 \end{aligned}\right.
\end{equation}
Lapenta (2017) \cite{lapenta_exactly_2017} shows that exact energy conservation is warranted under three conditions:
\begin{itemize}
    \item $\theta$ = 0.5
    \item The current density must be calculated using the average velocity, $\bar{\textbf{v}}_p$: \[\bar{\mathbf{J}}_g = \frac{1}{V_g} \sum_{p} q_p \bar{\textbf{v}}_p W(\textbf{x}_p^{n+\frac{1}{2}} - \textbf{x}_g )\]
    \item The discretised curl operator must preserve the following property of the continuous curl operator:
    \[\nabla \cdot (\textbf{E} \times \textbf{B}) = \textbf{B} \cdot (\nabla \times \textbf{E}) - \textbf{E} \cdot (\nabla \times \textbf{B})\]
\end{itemize}
We provide our own derivation in Appendix \ref{app: regular grid}, aiming to offer further insight into this proof.
The discretisation used in the field solver gives a large, sparse, block diagonal system which must be solved during each time step. The field solver is second-order accurate in time; the updated values for the fields are calculated between time points, while the used values in the time derivative are at the time points. We use the same spatial discretisation of the curl operator as in the original paper. For a one-dimensional spatial domain, this gives
\begin{align}
    \nabla_g \times \mathbf{E}^{n + \theta} &= \frac{1}{\Delta x}\left[\begin{matrix}
        0  \\
        -(E_{z,g+1} - E_{z,g})\\
            (E_{y,g+1} - E_{y,g})\\
    \end{matrix}\right]\\
    \nabla_g \times \mathbf{B}^{n + \theta} &= \frac{1}{\Delta x}\left[\begin{matrix}
        0  \\
        -(B_{z,g+1} - B_{z,g})\\
            (B_{y,g+1} - B_{y,g})\\
    \end{matrix}\right]\\
\end{align}

As the magnetic and electric fields are stored at the cell centres and vertices, respectively, this discretisation is also second-order. The calculated spatial finite difference value is used in the middle of the two used values at the spatial position of the temporal finite difference. This makes the simple first-order finite difference, in effect, a centred difference. 

The attentive reader may have noticed that the magnetic field term in the particle mover is not computed at the time $t_{n+\theta}$. As the magnetic field does not perform any work on the particles, the energy of the system is unaffected by this approximation. As a result, however, ECSIM can rewrite the particle mover as a direct update rule. It does this by defining a rotation matrix $\alpha_p^n$ which allows $\bar{\textbf{v}}_p$ to be written as
\[\bar{\textbf{v}}_p = \hat{\textbf{v}}_p + \frac{q_p \Delta t}{2 m_p}\hat{\textbf{E}}_p\]
 where 
 \[ \hat{\textbf{v}}_p = \alpha^n_p \textbf{v}_p^n, \quad \hat{\textbf{E}}_p = \alpha^n_p \textbf{E}_p^{n+\theta}
 \]
 This allows ECSIM to calculate the current density $\bar{\textbf{J}}_g$ exactly, without needing an iteration scheme, usually required due to the non-linear coupling.
 \[\bar{\textbf{J}}_g = \frac{1}{V_g}\left(\sum_p q_p \hat{\textbf{v}}_p W_{pg} + \frac{q_p \Delta t}{2 m_p}\sum_{g'}M_{g g'} \textbf{E}^{n+\theta}_{g'}\right)\]
 where $V_g$ is the volume of the grid cell $g$ and the matrices $M_{g g'}$ are called the \textbf{mass matrices}. The elements of these matrices, $M_{g g'}$, are given by
 \begin{equation}
     M_{g g'}^{ij} = \frac{\beta}{V_g}\sum_p q_p \alpha^{ij,n}_p W_{p g'} W_{p g}
 \end{equation}

 Lapenta (2023) \cite{lapenta_advances_2023} shows that the exact energy-conservation of ECSIM is compatible with subcycling.


%%% Local Variables: 
%%% mode: latex
%%% TeX-master: "thesis"
%%% End: 

\chapter{Parallel-in-Time methods}
\label{cha: pint}
This chapter investigates parallel-in-time methods. The focus is on parareal, a general scheme to parallelise the simulation of differential equations in time. This chapter first gives some background on the origin of parallel-in-time methods before reporting on parareal. Desired properties such as speedup and convergence are investigated, and section \ref{sec: parareal improvements} describes other parallel-in-time methods with improved properties.
\section{Background}
\label{sec: pint background}
Current chip and computer design tends more and more towards parallel processing instead of increased clock frequency \cite{bautista_intel_nodate}. A couple of reasons holding back increased clock frequency are: 
\begin{itemize}
    \item increasing power consumption
    \item increasing heat generation
    \item memory access is a bottleneck for faster calculations
\end{itemize}
Therefore, modern chips and systems incorporate more processors instead of making a single, more powerful version. 
Algorithms, however, need to take advantage of these extra processors. Solving differential equations has been extensively parallelised in the spatial domain \cite{adams_parallel_1999,du_expandable_2020}. Unfortunately, the parallel improvements are not infinite. Using more processors slows the calculations at a certain point due to the increasing communication overhead. Using extra cores to calculate solutions with increased spatial accuracy also has problems. For example, methods subject to the CFL condition would have to increase the accuracy in the time domain. This leads to more time steps that need to be simulated, possibly increasing the computational complexity. Most methods also have separate spatial and temporal accuracy. Even if the error in the spatial domain were to go to zero, the time domain would dominate the perceived error on the solution. These reasons should demonstrate that it is desirable to also parallelise differential equations in the time domain. 

The following sections assume an ODE as in equation \ref{eq: ODE}, dependent on time and simulated on $[0, T]$.
This range can be discretized into $N + 1$ time points $0 = t_0 < t_1 < t_2 < \hdots < t_{N} = T$. A constant distance between time points is assumed for the rest of this work, $\Delta t = t_{i+1} - t_i = \frac{T}{N}$.

\section{Parareal}
\label{sec: parareal}
A well-known parallel-in-time method is parareal \cite{lions_resolution_2001,gander_analysis_2014,d_samaddar_parallelization_2010, bal_symplectic_2008}. It resembles a standard predictor-corrector scheme, where a rough prediction of the solution is iteratively improved. Its convergence properties have been studied and it has been found to be best suited for parabolic PDEs \cite{gander_analysis_2007}. Unfortunately, the convergence for the hyperbolic case can be a lot less impressive, although even chaotic systems have been solved using it \cite{d_samaddar_parallelization_2010}. The algorithm uses a coarse solver, $\textbf{G}$, and a fine solver, $\textbf{F}$. These solvers use the known solution at time $t_n$ to calculate the solution at time $t_{n+1}$ 
\[\textbf{G}(\textbf{U}_n, t_n, t_{n+1}) = \textbf{U}_{n+1}, \quad
\textbf{F}(\tilde{\textbf{U}}_n, t_n, t_{n+1}) = \tilde{\textbf{U}}_{n+1}
\]
In practice, $\textbf{F}$ is the method of choice to solve the ODE, while $\textbf{G}$ is a computationally cheap approximator of the solution. Parareal hopes to calculate a solution with accuracy close to $\textbf{F}$ in a faster manner than a serial simulation. It does so by iteratively adapting an initial solution using the following predictor-corrector scheme:
\begin{equation}
\textbf{U}_{n+1}^{k+1} = \textbf{G}(\textbf{U}_n^{k+1}, t_n, t_{n+1}) + \textbf{F}(\textbf{U}_n^k, t_n, t_{n+1}) - \textbf{G}(\textbf{U}_n^k, t_n, t_{n+1})
\end{equation}
This iteration scheme is repeated until the solution has converged. 
Intuitively, it is clear that the required number of iterations, $K$, should be small for a large speedup. The required $K$ can be reduced by obtaining a good approximation for the initial condition. In practice, this is often done by performing a sequential solve of the system using the coarse propagator. These initial conditions are then also used for the $\textbf{G}(\textbf{U}_n^k, t_n, t_{n+1})$ term in the first iteration. 

\subsection{Speedup and Parallel Efficiency}
The speedup and parallel efficiency equations for parareal demonstrate the importance of reducing the number of iterations. We represent the cost of communication between processors by $C_\mathrm{comm}$. In distributed memory systems, communication could consist of scattering before the parallel section and gathering afterwards. For shared memory systems, communication would be the accessing of memory.
The computational cost of performing $\textbf{F}(\textbf{U}_n, t_n, t_{n+1})$ is represented as $\gamma_F$, while the time cost of the coarse version $\textbf{G}(\textbf{U}_n^k, t_n, t_{n+1})$ is $\gamma_G$. Assuming $p$ processors are used, the time required for the serial and parareal solutions can be described as
\begin{equation}
\begin{split}
    T_{\mathrm{serial}} &= N \gamma_F\\
    T_{\mathrm{parareal}} &= N \gamma_G + K\left[\frac{N}{p}\gamma_F + C_\mathrm{comm}\right] + K N \gamma_G\\
\end{split}
\end{equation}
The time required for parareal can be split into three parts: the initialisation, a parallel section, and a serial section. 
The speedup is given as the ratio between the serial and parallel time.
\begin{equation}
\begin{split}
\label{eq: speedup}
    S &= \frac{T_{\mathrm{serial}}}{T_{\mathrm{parareal}}}\\
     &= \frac{1}{K \left[\frac{1}{p} + \frac{1}{N \gamma_F} C_\mathrm{comm} + (\frac{1}{K} + 1)\frac{\gamma_G}{\gamma_F}\right]}
\end{split}
\end{equation}
While the speedup is often used as a metric, it can misrepresent the performance. For example, a speedup of $2$ on a machine with two cores is good, while the same speedup on $100$ processors is quite bad. A more informative property is the parallel efficiency, the ratio between the speedup and the used number of cores. 
\begin{equation}
\begin{split}
\label{eq: parallel efficiency}
   E &= \frac{S}{p}\\
    &= \frac{1}{K\left[1+\frac{p}{N \gamma_F} C_\mathrm{comm} + p(\frac{1}{K} + 1)\frac{\gamma_G}{\gamma_F}\right]}
\end{split}
\end{equation}
This formula gives a quantitative meaning to the intuition behind parareal. On one hand, the fine integrator should be expensive, while the coarse solver should be as cheap as possible. However, the efficiency will always be bounded by $\frac{1}{K}$. Thus, any change in solvers that increases the required iterations should be investigated for actual improvements in parallel efficiency. We note that the cost of initialisation is usually amortized by the number of iterations and thus will penalize the speedup when a small number of iterations is achieved.

\subsection{Stopping Condition}
As parareal is an iterative method, a stopping condition must be defined to decide when the solution has converged. If the number of parareal iterations is equal to the number of coarse time steps, the fine integrator will have propagated its solution throughout the entire time domain. Due to this, the maximal amount of iterations needed is bounded by the number of time steps. However, as stated before, this would give abysmal speedup and should thus be avoided. There are multiple choices as to how convergence is tested. In this thesis, convergence is assumed when the relative state changes are smaller than some tolerance $\epsilon_{tol}$
\[\max_{n=0...N}\left|\frac{\textbf{U}^{k+1}_n -\textbf{U}^{k}_n}{\textbf{U}^{k+1}_n} \right| < \epsilon_{tol}\]
In the context of energy conservation one could also decide to assume convergence when the relative error on the energy is below the given tolerance. We show in chapter \ref{cha: results} that the error on the energy is bounded by the state error, making the latter a better test. As a result, we decide to test against the error on the state variables.

\subsection{Energy Conservation}
It is known that parareal does not conserve energy exactly for arbitrary length time intervals even if energy conserving methods are chosen for the coarse and fine integrator \cite{gander_analysis_2014}. As a result we do not concern ourselves with exact energy conservation over the time interval when using parareal. We do, however, show that bounds can be set on the error of the energy.

\subsection{Time Scaling}
Assume a time interval must be simulated $t \in [0, T]$ with a given coarse time step size $\Delta t$. In the best case scenario, the number of available cores $p$ is equal to the fraction $\frac{T}{\Delta t}$. This will ensure that each call to the fine solver can be executed in full parallelism. In systems with fewer resources, however, it might occur that the fraction $\frac{T}{\Delta t}$ is larger than the available number of cores. In such cases, there are two main strategies to perform the parareal simulation. One can perform one parareal solve on the larger time frame, meaning processors will compute multiple fine integrator steps in serial, thus reducing the speedup in the parallel section. On the other hand, it is also possible to split up the time interval into $m$ chunks, for which $\frac{T}{m \Delta t} \le p$. Parareal is then performed multiple times in serial on each chunk. When to use which strategy is highly dependent on the discrepancy between $p$ and $\frac{T}{\Delta t}$ and the convergence of parareal on the given problem. D. Sammadar et al. (2010) \cite{d_samaddar_parallelization_2010} found that parareal has a convergence behaviour dependent on the iteration number. The number of time steps that converge each parareal iteration increases along with the iteration number. Typically, only one time step converges each iteration when the number of iterations is small, reminiscent of a ``booting up'' sequence. D. Sammadar et al. found the number of converging time steps to rise linearly until a maximum is obtained. This behaviour suggests that restarting the parareal algorithm multiple times on smaller time intervals might thus be harmful to the overall computational runtime. 

\section{Improvements}
\label{sec: parareal improvements}
The fame, along with the apparent shortcomings of parareal, have led to improvements in the method. We briefly introduce two methods that improve on the traditional parareal algorithm in different aspects. We also list why we do not use them in the current thesis.
\subsection{PFASST}
\label{subsec: intro pfasst}
The first method is the Parallel Full Approximation Scheme in Space and Time \cite{emmett_toward_2012} method, PFASST. It encapsulates parallelisation in both time and space for maximal speedup. Instead of accepting any black box time integrators $\textbf{G}$ and $\textbf{F}$, it requires the use of Spectral Deferred Corrections (SDC) \cite{dutt_spectral_2000}, for its solvers. SDC is a spectral predictor-corrector scheme, where the integral of the differential equation is iteratively approximated using injected Gaussian nodes. The SDC iterations performed in the coarse integrator can actually speed up and improve the fine SDC integration through the use of the Full Approximation Scheme (FAS) \cite{brandt_multi-level_1976}. FAS is a multigrid method which is used to correct solutions at fine scales based on calculated errors on coarser scales \cite{henson_multigrid_2003}.

% The integration of the differential equation \ref{eq: ODE} can be written using the following Picard equation:\[
% 	\textbf{u}(t) = \textbf{u}_0 + \int^t_0 \textbf{f}(\tau,\textbf{u}(\tau))\tau
% \]
% Assuming an approximation of the real solution $\textbf{u}^k(t)$ is given, the error, $\delta^k(t)$, and residual, $\varepsilon^k(t)$, can be defined as:\[
% 	\delta^k(t) = \textbf{u}(t) - \textbf{u}^k(t), \quad
% 	\varepsilon^k(t) = \textbf{u}_0 + \int^t_0 \textbf{f}(\tau,\textbf{u}^k(\tau))\tau - \textbf{u}^k(t)
% \]
% SDC solves the initial differential equation by injecting $M$ Gaussian nodes in each of the $N$ time sections and then iteratively improving $\textbf{u}^k(t)$. Essentially, it solves $N$ IVPs in succession. This leads to continuous approximations, $\textbf{u}^{k}(t)$, on each section after every iteration, where the continuous approximations are always Lagrange polynomials through the Gaussian nodes. The updates are performed by approximating the residual and using this to calculate the error, which then gets subtracted from the approximate solution. 

% As the name suggests it also uses the Full Approximation Scheme, FAS \cite{brandt_multi-level_1976}.
% \newline
% Assume an approximation $v^f$ is found on a fine scale, so that $u^f = v^f + e^f$, where $u^f$ solves:
% \[A^f(u^f) = g^f\]
% Taking the coarse problem to be equal to:
% \[A^c(v^c + e^c)  - A^c(v^c) = r^c\]
% where $r^c = I_f^c(r^f) = I_f^c(g^f - A^f(v^f))$ is the residual of the coarsened problem. If $v^c$ is then also defined as the coarsened $v^f$, then the substituted coarse residual can be rewritten as:\[
% A^c(I_f^c(v^f) + e^c) = A^c(I_f^c(v^f)) + I_f^c(g^f - A^f(v^f))\]
% The right hand side can be calculated and the resulting system can be solved to find a solution $u^c$. If the error of this solution is calculated as $e^c = u^c - I_f^c(v^f)$, then the interpolated error can be used to update the fine grid approximate solution $v^f \leftarrow v^f + I_c^f(e^c)$
% \cite{henson_multigrid_2003}
% \newline
% FAS allows PFASST to use the coarse solver to calculate better and more efficient fine SDC solver solutions. 

The PFASST algorithm improves the parallel efficiency of parareal. Its parallel efficiency is bounded by the ratio between the iterations needed by the SDC solver against the number of parareal iterations, $\frac{K_{SDC}}{K_{parareal}}$. While the theoretical parallel efficiency of PFASST can be better than the traditional parareal implementation, it requires the use of the SDC method in the used fine and coarse integrators. As ECSIM instead uses one-step methods based on finite differences, the base ECSIM method would have to be adapted to use the SDC method. This was considered to be outside of the scope of this thesis.

\subsection{Symplectic Parareal}
\label{subsec: intro symplectic parareal}
The symplectic parareal method improves on the traditional parareal algorithm by conserving the symplectic structure of PDEs \cite{bal_symplectic_2008}. If the exact solution $u(t + \Delta t)$ can be found as $g(u(t),t)$, then the original parareal scheme uses a discretized version $g_{\Delta}$ to perform the following predictor-corrector scheme $g = g_\Delta + (g - g_\Delta)$ The symplectic version, instead, assumes the following solution $f = \psi_\Delta \circ f_\Delta$. It uses the property of symplectic maps that the composition of symplectic maps is also symplectic. 
% So if both $f_\Delta$ and $\psi_\Delta$ are symplectic, the parareal iteration will be too. The function $\psi_\Delta$ is practically implemented through an interpolation scheme, where extra care is taken to make it symplectic. This is done using \textit{generating functions} \cite{hairer_geometric_2006}.  $\psi_\Delta$ is actually an approximation of the identity function of order one higher than the order of $f_{\Delta}$. Due to this \cite{hairer_geometric_2006}, a generating function can be found of the form $S(\textbf{q}^*, \textbf{p}) = \textbf{q}^* \textbf{p} + \gamma(\textbf{q}^*, \textbf{p})$, where it is assumed that $u(t) = (\textbf{q},\textbf{p})$, $(\textbf{q}^*, \textbf{p}^*) = \psi_\Delta(\textbf{q},\textbf{p})$ and $\gamma$ is a mapping from $\mathbb{R}^{2d}$ to $\mathbb{R}$. If the following equations are satisfied:
% \[\textbf{q}^* = \textbf{q} - \diffp{\gamma}{{\textbf{p}}}(\textbf{q}^*, \textbf{p}), \quad 
% 	\textbf{p}^* = \textbf{p} + \diffp{\gamma}{{{\textbf{q}^{*}}}}(\textbf{q}^*, \textbf{p})\]
% Then an interpolation of $\gamma$ through the $N$ points $(f_\Delta(U^1_n),\psi_\Delta(f_\Delta(U^1_n)))$ will be symplectic. If enough points are used, then this interpolation will also serve as an interpolation of $\psi_\Delta$\cite{bal_symplectic_2008}.

The symplectic structure can help to bound the error on energy on longer time frames. In a conversation with Dr. Martin Gander during the PinT 2024 Workshop on February 8, 2024, it was mentioned that the symplectic parareal version is very expensive to calculate. Due to this, large speedup is hard to obtain. As we do not concern ourselves with exact energy conservation, we decide not to use this more expensive version of parareal. We show in Chapter \ref{cha: results} that reasonable bounds on the energy error can also be obtained using tolerances on the state variables.


%%% Local Variables: 
%%% mode: latex
%%% TeX-master: "thesis"
%%% End: 

\chapter{Methodology}
\label{cha: methodology}

%%% ============================================================================================ %%%

We now investigate the performance of the parareal method applied to PIC simulations using a simplified model of ECSIM. This simplified model only takes one dimension into account for the spatial discretisation. However, the velocity and field vectors at each position are tracked in either one dimension (1D1V) or three dimensions (1D3V). This enables the simulation of complex phenomena while reducing computational complexity.

Firstly, the efficacy of parareal is tested using a well-known implicit method: the Crank--Nicolson (CN) integrator \cite{Crank_Nicolson_1947}. We apply this method to the one-dimensional diffusion equation and show that CN indeed achieves second-order accurate solutions, with and without parareal. The performance of the ECSIM solver with parareal is the main point of interest for this project. To test the validity of our code implementation of the ECSIM algorithm, unit tests are performed, and we compare our simulation results with those presented by Lapenta in 2017 and 2023 \cite{lapenta_exactly_2017,lapenta_advances_2023}. After confirming a correct implementation, we combine ECSIM with parareal to test its performance. We quantify the efficacy of the parareal algorithm applied to a method using the following metrics:
 \begin{itemize}
    
    \item \textbf{Accuracy}, calculated as the error of the parareal solution compared to the serial solution of the fine integrator
    
    \item \textbf{Speedup}, defined as the ratio of time taken by the serial solve to that of the parareal solve $\left(\frac{T_\mathrm{serial}}{T_\mathrm{parareal}}\right)$
    
    \item \textbf{Parallel efficiency}, defined as the ratio of the speedup obtained to the number of cores used $\left(\frac{\text{speedup}}{\text{number of cores}}\right)$
    
    \item \textbf{Computational runtime}, the simulation time elapsed
 
 \end{itemize}
We analyse these aspects of the parareal algorithm using the following set of test cases:
\begin{itemize}

    \item CASE I: Diffusion equation

    \item CASE II: Decoupled manufactured solutions

    \item CASE III: Two-stream instability, an electrostatic toy problem

    \item CASE IV: Weibel or transverse beam instability, an electromagnetic toy problem
    \end{itemize}

In this chapter, we discuss the different test cases and implementation details. The following chapter contains the results of the experiments.

\section{Test Cases}
\subsection{CASE I: Diffusion Equation}
The first test case that will be used during the experiments is the one-dimensional diffusion equation
\begin{equation}
    \frac{\partial u(x,t)}{\partial t} = \frac{\partial^2 u(x, t)}{\partial x} \quad \text{with initial condition} \quad u_0(x) = u(x,0)
\end{equation}
where $u(x,t)$ is the state variable at position $x$ and time $t$. The spatial domain is split up into $N_x$ grid cells. The spatial derivative is discretised using the standard second-order centred finite difference scheme, and periodic boundary conditions are used on the spatial domain $x \in [0,2\pi[$. The diffusion equation is a parabolic PDE and has been shown to have good convergence properties for parareal \cite{gander_analysis_2007}, especially when A-stable methods are used.

\subsection{CASE II: Decoupled Manufactured Solutions}
The second test case is used to show the correctness of the implementation of the ECSIM particle mover and field solver separately. Due to interactions between the particle mover and the field solver, the order of the complete ECSIM method can be different from the order of the particle mover and field solver on their own \cite{lapenta_exactly_2017}. As a result, we only show the second-order convergence of the field solver and particle mover without any coupling between them. This is done using two manufactured solutions, where the initial conditions and inputs are designed so that a desired analytical solution will be approximated. This test case involves a 1D3V domain, meaning the spatial domain is one-dimensional, while the velocity of each particle and the electric and magnetic field vectors at each grid cell are three-dimensional. When advancing the position, only the first dimension of the velocity is used.

The particle mover is tested by only considering one particle which is influenced by given external fields. The desired velocity in function of time is chosen as \[
    \textbf{v}(t) = \left[\begin{matrix}
    \cos(t)\\
    -\sin(t)\\
    \cos(t) - \sin(t)
    \end{matrix}\right]
\]
To obtain such a solution for the velocity the position, electric field and magnetic field can be chosen as.
\[
    x(t) = \sin(t), \quad
    \textbf{E}(t) = \left[\begin{matrix}
    -\frac{m}{q}\sin(t) + \frac{m}{2q}\cos(t)\\
    -\frac{m}{2q}\sin(t)\\
    -\frac{m}{2q}\sin(t) - \frac{3m}{2q}\cos(t)
    \end{matrix}\right], \quad \textbf{B}(t) = \left[\begin{matrix}
    -\frac{m}{2q}\\
    \frac{m}{2q}\\
    \frac{m}{2q}
    \end{matrix}\right]
\]
Note that the magnetic field is stationary in time, allows us to achieve second-order accuracy in time, otherwise the coupling with the magnetic field would break down the method to first-order.

The field solver is also tested using a manufactured solution. We define an analytical solution to the Maxwell equations and set the initial conditions equal to the value obtained by the analytical solution at $t=0$.
\[    \textbf{E}(x,t) = \left[\begin{matrix}
    0\\
    \cos(\omega t)\sin(k x)\\
    0
    \end{matrix}\right], \quad \textbf{B}(x,t) = \left[\begin{matrix}
    0\\
    0\\
    -\sin(\omega t)\cos(k x)
    \end{matrix}\right]
\]
These equations are valid if $k = \omega$; in our implementation we choose to set these to $\omega = k = 3$.

We close our description of this test case with the statement that special care must be taken when sampling the analytical solution. The different discretised equations use different grid points in the temporal and spatial domain. For example, the position is calculated between time points $t_{n+\frac{1}{2}}$, while the velocity is computed at each time point $t_n$. In the spatial domain, the electric field is tracked on the grid cell vertices, while the magnetic field is stored on the grid cell centres. If the values are not sampled at the correct point in the space-time domain, the accuracy of the method might break down.


\subsection{CASE III: Two-Stream Instability}
The first of the two fully coupled plasma simulation test cases we use is the 1D1V two-stream instability. We assume an initially uniform and unmagnetised plasma, divided into two Maxwellian counter-streaming beams. The two-stream instability arises when more particles are faster than the phase velocity of the electric field wave than there are slower particles. The wave will absorb energy from the particles, exponentially growing the wave, and the streams will mix. In our setup, the base velocity of these beams is $\frac{|v_0|}{c} = 0.2$, and the thermal velocity is equal to $\frac{v_th}{c} = 0.01$. We then offset the velocity with a small perturbation to excite the instability of the fastest growing mode in our setup $k = 3$
$\delta v = \frac{v_{th}}{10}\sin(\frac{2\pi}{L} m \textbf{x}_p)$
\cite{chen_introduction_1984}. We note that this test case cannot fully test the magnetic field coupling as it does not influence the particles or electric field in the 1D1V case. Figure \ref{fig: two-stream instability} shows the used initial velocity distribution of the two-stream instability.
\begin{figure}
    \centering
    \includegraphics[width=0.7\linewidth]{images/two-stream-distr.png}
    \caption{Particle velocity distribution function of the two-stream instability.}
    \label{fig: two-stream instability}
\end{figure}

\subsection{CASE IV: Weibel Instability}
The final test case is a full 1D3V transverse beam or Weibel instability. Here, the magnetic field plays a large role in the instability. It arises for two counter-streaming beams with a velocity anisotropy in one direction. We again assume an initially uniform and unmagnetized plasma, where the $y$-direction velocity is higher than the other directions $x$ and $z$. The base velocity of our beams is $\frac{|v_0|}{c} = 0.8$ in the $y$-direction, and the thermal velocity is equal to $\frac{v_th}{c} = 0.01$ for all directions. In this scenario, a magnetic field that arises in the $x$ or $z$ direction, e.g. $\textbf{B} = B_z \cos(k x)$,  would create current sheets $\textbf{j} = q_p n \textbf{v}_p$ that are phased to generate the magnetic field creating the shape. This leads to instability as the magnetic field grows and grows \cite{chen_introduction_1984}. This behaviour is shown in Figure \ref{fig: weibel instability}, where the deflection of the particles is visualized.
\begin{figure}
    \centering
    \includegraphics[width=0.7\linewidth]{images/weibel_mechanism.png}
    \caption{Physical mechanism of the Weibel instability \cite{chen_introduction_1984}.}
    \label{fig: weibel instability}
\end{figure}


\section{Implementation Details}
The developed ECSIM code only depends on the standard \texttt{C++} libraries and the publicly available \texttt{Eigen} library \cite{gael_guennebaud_and_benoit_jacob_and_others_eigen_2010}. The code is parallelised using \texttt{OpenMP}, which supports shared-memory parallelisation and GPU-offloading (from version  4.0 onward). While GPU-offloading is beyond the scope of this work, one could easily adapt this code by replacing the \texttt{Eigen} matrices with a GPU-supported linear algebra library. The code developed within the framework of this project is open-source and can be obtained from \url{https://github.com/BramLeys/ecsim-parareal/tree/main}. 

Using \texttt{OpenMP} instead of \texttt{MPI} leads to some implementation choices for parareal. Parareal consists of a serial part, where the coarse integrators and updates are performed, and a parallel section, where the fine solver is computed. This means there are idle cores during the serial section that can be used for other calculations. When the parallelisation is performed using \texttt{MPI}, cores are often employed in a pipelining scheme to minimise the number of idle cores. In this framework, after a time step is updated, it is immediately sent to a different core to be simulated using the fine solver instead of waiting until the new values for all time steps are computed. As we are using \texttt{OpenMP}, we will instead attempt to use these idle cores to improve the accuracy of the coarse solver while using parallelisation to counter the increased computational cost (see section \ref{sub: subcycling} for details). \texttt{OpenMP} is designed for shared-memory systems, and as such, all tests are performed on one node. 
Since GPUs have a vast amount of computational cores on one node, this would allow for more cores while still being able to make optimal use of \texttt{OpenMP}. To use multiple nodes, \texttt{MPI} would be preferred to efficiently use the available cores.
 The tests were performed on two systems: an MSI GS65 Stealth 9SF with an Intel i7-9750H CPU at a base clock speed of 2.60GHz and 32 GB of RAM and the Tier-2 wICE system at the Flemish Supercomputing Center (VSC). The experiments were run on the Sapphire Rapids nodes, each with 2 Intel Xeon Platinum 8468 CPUs, with a total of 96 cores running at 2.10GHz base clock. Each node has 256 GiB of RAM. The preliminary tests were performed on the MSI GS65, while time-sensitive experiments, such as the speedup and performance tests, were performed on the VSC. 

\section{Linear Solvers}
ECSIM must solve a linear system of equations during the field solver step to update the electric and magnetic fields in accordance with Maxwell's equations. In the next chapter, we explore different solvers for use in the coarse and fine integrators of parareal. Here, we present potential methods for solving this system, making a distinction between direct and iterative solvers.

Direct solvers have the advantage of computing the solution exactly. However, they can be computationally prohibitive for large systems. In a serial context, the solution to the linear system generated in ECSIM should be accurate up to machine precision, ensuring exact energy conservation. Since the basic parareal algorithm does not inherently conserve this property \cite{gander_analysis_2014}, it must iterate to very stringent tolerances to achieve machine precision energy conservation. As total energy conservation is not a requirement for this thesis, we have more flexibility in choosing methods to solve this linear system. This allows us to consider a broader range of solvers beyond those that achieve machine precision.

The matrix resulting from the linear system is sparse and has a block-diagonal structure, which lends itself to iterative solvers, such as the generalised minimal residual method (GMRES). The discretisation matrix is not symmetric, invalidating specific solvers, such as Cholesky-based decompositions. 

We now outline the three methods used in the further chapters: sparse lower upper decomposition (sparse LU decomposition), GMRES \cite{youcef_saad_martin_h_schultz_gmres_1986} and biconjugate gradient stabilised method (BiCGSTAB) \cite{van_der_vorst_bi-cgstab_1992}.

\subsubsection{Sparse Lower Upper Decomposition}
LU decomposition is a direct method that factorises a matrix $\textbf{A}$ into a lower triangular matrix $\textbf{L}$ and upper triangular $\textbf{U}$ such that $\textbf{A} = \textbf{LU}$. To ensure a proper decomposition for any square matrix, it might be necessary to perform a permutation of rows so that there are no 0 elements on the diagonal of $\textbf{A}$ at each step. This is called LU decomposition with partial pivoting. This leads to $\textbf{PA} = \textbf{LU}$. In the sparse case, this permutation matrix $\textbf{P}$ is also chosen to minimise fill-in. Fill-in is the occurrence of a non-zero in the $\textbf{L}$ or $\textbf{U}$ matrices, while the element at the same place in the $\textbf{A}$ matrix is zero. The factorisation matrices can cheaply compute the solution to the linear system using forward and backward substitution.
\begin{align}
    \textbf{Ax} &= \textbf{b}\\
    \textbf{LUx}&= \textbf{b}\\
    \textbf{Ly} = \textbf{b} \quad &\quad \textbf{Ux} = \textbf{y}
\end{align}

\subsubsection{Generalised Minimal Residual Method}
GMRES is an iterative linear solver based on Krylov subspaces, starting from a given initial guess $\textbf{x}_0$. The n-th Krylov subspace can be written as $K_n = \mathrm{span}(\{\textbf{A}\textbf{x}_0-\textbf{b}, \textbf{A}(\textbf{A}\textbf{x}_0-\textbf{b}),...,\textbf{A}^{n-1}(\textbf{A}\textbf{x}_0-\textbf{b})\})$. The solution $\textbf{x}$ is approximated at iteration n by $\textbf{x}_n = \textbf{x}_0 + \textbf{Q}_n \textbf{y}_n$, where $\textbf{Q}_n$ is an orthogonal basis for $K_n$ and $\textbf{y}_n$ are the coefficients that minimise the error $\|\textbf{x}-\textbf{x}_n\|$. GMRES can suffer from storage and computational issues as the dimension of the Krylov subspace increases. GMRES must store the Hessenberg matrix, which contains all the orthogonal vectors forming the orthogonal basis. Additionally, it has to orthogonalise each new vector against all previously computed vectors, which can become quite costly. This can be remedied by ``restarting'' the algorithm. This involves starting a new GMRES algorithm using the previously found solution as the initial guess. The method is often used to solve large sparse systems and can be applied to any nonsingular square matrix\cite{he_parallel_2023}.


\subsubsection{Biconjugate Gradient Stabilized Method}
BiCGSTAB is another iterative method. As the name suggests, it is based on the biconjugate gradients method, which is a generalisation of the conjugate gradients method. Like the Cholesky decomposition-based methods, the conjugate gradient method does not apply to non-symmetric matrices. The biconjugate gradient method can solve non-symmetric systems; however, it is numerically unstable, leading to the use of the BiCGSTAB in most practical applications. The BiCGSTAB algorithm performs two BiCG steps followed by a stabilisation step. BiCG uses a biorthogonal system, needing to keep track of two separate residuals, which can lead to unstable behaviour. The BiCGSTAB method uses both a direction and a magnitude parameter, defined by minimising the resulting residuals like GMRES, to smooth convergence and stabilise the method. It is also often used for large sparse matrices \cite{yang_improved_2002,krasnopolsky_revisiting_2020} and can converge faster than GMRES. It cannot restart the algorithm like GMRES, although the memory requirements should be smaller than for GMRES. 
\chapter{Parareal Solvers for Particle-in-Cell Simulations}
\label{cha: results}

%%% ============================================================================================ %%%

This chapter lists the results and conclusions of our experiments on the test cases described in Chapter \ref{cha: methodology}. 

We investigate the accuracy and correctness of the CN, ECSIM and the parareal algorithm implementations. These tests are primarily based on convergence analysis. 
Once the correctness of our implementation can be assumed, we perform parameter studies to investigate the computational runtime and speedup for different coarse and fine propagators. 
 The following parameter studies are considered: 
 \begin{itemize}
 
    \item Temporal coarsening
    
    \item Scaling tests on shared-memory architecture
    
    \item Linear solvers (used during ECSIM)
 
 \end{itemize}
A spatial coarsening test was also considered. However, elementary tests revealed that studying the efficacy of spatial coarsening would require a comprehensive analysis with various high-order polynomial interpolation methods, which is beyond this project's scope. 

Note that parareal solvers are known to exhibit poor conservation of energy \cite{gander_analysis_2014}. Energy conservation is shown during serial solutions to assess the correctness of the implementation. We do not concern ourselves with the exact energy-conserving property of ECSIM when using parareal. However, the error incurred on the energy should be bounded even for parareal, which our results show can be achieved.

\section{Validation of Implementations}
\subsection{Crank--Nicolson}
\label{sec: cn}
We start our analysis of the parareal algorithm using the well-known second-order implicit CN integrator, on CASE I. 
As a first test, consider a simulation over $T = 1$\,second discretised in $N_t$ time points, with a grid of size $N_x = 100$. We denote the time step and grid cell sizes as $\Delta t$ and $\Delta x$, respectively. We first set the initial value of each grid cell to a random value in [-1, 1].  This may seem strange, considering normal tests involve smooth initial conditions. However, these conditions show certain properties that are also observed in the test cases for ECSIM. The solution of the discretised system at time step $t_n$ is denoted as $\textbf{U}_n$. 
Figure \ref{fig: CN-convergence-random} shows the serial and parareal simulation errors compared to a reference solution with decreasing time step size. For parareal, this means reducing the time step size of the fine solver while keeping the coarse solver time steps at the original size ($\Delta t_\mathrm{Coarse}= 10^{-2}$). The reference solution, $\tilde{\textbf{U}}_{n}$, is calculated in serial using a time step $50$ times smaller than the smallest time step used for the convergence analysis. This should make the reference solution accurate enough to make meaningful conclusions about the convergence of the error compared to an exact solution. The error is computed as the relative l2 norm of the difference at $t=1$\,s:
\[\mathrm{error} = \frac{\|\tilde{\textbf{U}}_{N_t} - \textbf{U}_{N_t}\|_2}{\| \tilde{\textbf{U}}_{N_t} \|_2}\]
\begin{figure}[h]
    \centering
    \includegraphics[width=1\linewidth]{figures/eps/CNConvergence.eps}
    \caption{Error obtained using Crank--Nicolson in serial and using parareal for CASE I with random initial conditions and decreasing time step sizes. The error is computed against a reference solution calculated using a times step size 50 times smaller than the smallest step size shown.}
    \label{fig: CN-convergence-random}
\end{figure}
Figure \ref{fig: CN-convergence-random} shows the expected second-order convergence for the errors. The errors incurred by the parareal implementation are almost indistinguishable from the serial solution, indicating that the algorithm does approximate the fine solution after convergence. Note that the difference between parareal and serial would become visible if the error of the serial solution becomes smaller than the tolerance used to define the convergence criteria of parareal. 
Although Figure \ref{fig: CN-convergence-random} shows the desired order of accuracy for the solutions of the method using the parareal algorithm after the parareal algorithm, some concerns arise when looking at the convergence of parareal during its iterations. In the next section, we will see that the error estimation used in parareal can increase between iterations instead of converging to 0, leading to more iterations being needed and, as a result, negating any speedup.

\subsubsection{Parareal Convergence}
\label{sub: convergence parareal}
The parareal algorithm must estimate the error of each time step, $n$, compared to the serial solution during each iteration, $k$, to determine whether the algorithm has achieved the desired accuracy. Our implementation approximates this error by the relative state change between iterations:\[
E^k_n = \frac{\|\textbf{U}^k_n - \textbf{U}^{k-1}_n\|_2}{\|\textbf{U}^k_n\|_2}
\]This thesis will refer to this error as the parareal error from here on. The convergence of the parareal algorithm can thus be investigated by looking at the evolution of these state changes. This reveals that, for the simulations performed for Figure \ref{fig: CN-convergence-random}, the difference in states between parareal iterations increases between successive iterations instead of converging to 0. This is quite unexpected considering the convergence theorems of parareal \cite{gander_analysis_2007}. CN is an A-stable method and is seemingly stable based on the convergence results of Figure \ref{fig: CN-convergence-random}. One would thus expect the error to decrease in an orderly fashion instead of increasing, regardless of the simulation parameters. 
The change between iterations only reaches 0 when the number of iterations equals the number of time steps. This is the expected behaviour since, at this point, the fine solver will have propagated throughout the entire time domain. However, when this scenario arises, speedup is impossible to achieve because the fine integrator has been performed once for every time point. 
This behaviour arises due to accuracy issues where the used CN method in the coarse solver does not have the exact expected second-order accuracy. This can be seen in Figure \ref{fig: CN-convergence-random}, the graph shows a different order at the time step size of the coarse solver ($\Delta t_\mathrm{Coarse} = 10^{-2}$). The discontinuous initial conditions cause this loss in accuracy. This is most likely due to the high frequencies contained in the noisy initial conditions. As CN is A-stable, it will not dampen these oscillations, which allows them to reduce the accuracy of the solution due to their highly oscillatory behaviour. We note that using an L-stable method, such as backward Euler, dampens such oscillations and was found to converge properly for the given test cases. However, the plasma test cases, CASE III and CASE IV, are instabilities. We do not want these to be damped and, as such, we do not use such L-stable methods. We find that decreasing the time step size or increasing the grid cell size can also improve the convergence. This indicates a CFL-like condition violation, where the ratio of the time step size and grid cell size are important. However, the observed behaviour is not typical of an actual CFL condition violation. Firstly, the behaviour can be avoided by not considering random initial conditions, while the CFL condition should be independent of the initial condition. For example, a smooth periodic sine wave as the initial condition ensures a converging trend. Secondly, the CFL condition is usually a requirement for stability. The method, however, is ``inaccurate'', not unstable as seen in \ref{fig: CN-convergence-random}. Unstable methods would give errors higher than  $10^{-2}$.

We show different techniques that can be employed to ensure converging behaviour in Figure \ref{fig: CN-parareal-convergence}, where the parareal error for five different situations is plotted. Note that the tolerance for the parareal iteration is set to $10^{-8}$. The original simulation using $\Delta t_\mathrm{Coarse} = 10^{-2}$ and $Nx = 100$ shows that the parareal algorithm does not properly converge. Only after a certain amount of iterations does the parareal error start decreasing. We note that the convergence eventually sets in and has been observed to eventually terminate the algorithm properly if enough time steps are considered. However, the other simulations show much better convergence properties and subsequent speedup. Especially, the smooth initial conditions show much faster convergence. However, the random initial conditions can also converge properly when using the correct discretisation. 
\begin{figure}[h]
    \centering
    \includegraphics[width=1\linewidth]{figures/eps/CN_parareal_Convergence.eps}
    \caption{Errors calculated during each parareal iteration when simulating CASE I, using different initial conditions and simulation parameters.}
    \label{fig: CN-parareal-convergence}
\end{figure}

\subsubsection{Tolerance of Parareal}
\label{sub: tol parareal}
We will now have a closer look at the tolerance scheme used for parareal. This tolerance defines the stopping condition for the iterative algorithm based on the currently estimated error.
The error incurred at the $k^\mathrm{th}$  parareal iteration compared to the fine solution is estimated by the parareal error. Classically, the parareal solution is considered converged if the error for each time step is lower than a given tolerance, $\varepsilon_\mathrm{tol}$. \[
    \max_{n =1...N_t}E^k_n < \varepsilon_\mathrm{tol}
\] 
Since convergence typically starts from the initial condition and propagates forward in time, the condition has been modified so states can converge individually \cite{d_samaddar_parallelization_2010}. This allows the algorithm to skip already converged states, thus reducing computations. In the standard parareal formulation, all states must be updated with each iteration. However, this leads to possibly wasted computations when most states are already converged since differences would not influence the solution much and might also introduce errors induced by resonance \cite{d_samaddar_parallelization_2010}. This method makes sense when the selected convergence tolerance is the same as the machine precision. Changes smaller than this tolerance are most likely due to rounding errors. However, this simple reasoning no longer holds when more lenient tolerances are used. It should, therefore, also be investigated whether this strategy of using the tolerance at a time step level does not impede the convergence of parareal. Figure \ref{fig: CN-parareal-convergence-tolerance} shows the parareal error and the error compared to the serial solutions for the two convergence strategies. Only the final iteration is different between the two strategies. However, the error estimate is always an upper bound for the actual error. This means that it is a good error estimator as it will ensure the error with regard to the serial solution is always smaller than the user-defined tolerance. They are also relatively close to each other, indicating that the parareal algorithm will not perform too many extra iterations using this estimated error than if it had access to the actual error compared to the fine solution.
\begin{figure}[h]
    \centering
    \includegraphics[width=1\linewidth]{figures/eps/Tolerance_parareal_check.eps}
    \caption{Errors of a simulation of CASE I using simulation level and time step level tolerance strategies for parareal. Left: Parareal error calculated during each parareal iteration. Right: Error of the parareal solution at each iteration compared to the serial fine solution.}
    \label{fig: CN-parareal-convergence-tolerance}
\end{figure}

\subsection{Energy Conserving Semi-Implicit Method}
We now demonstrate the correct implementation of the ECSIM algorithm. 
\subsubsection{Convergence Analysis}
\label{sub: convergence ecsim}
Firstly, we use CASE II to show that the uncoupled field solver and particle mover achieve second-order accuracy. As the particle mover is not dependent on the spatial grid discretisation when decoupled from the field solver, we only show second-order accuracy in the time domain. Figure \ref{fig: ecsim-convergence-posvel-time} shows the error of the particle mover on both the position and velocity for different time step sizes compared to the analytical solution. It shows that the error decreases as a second-order accurate method.

\begin{figure}[h]
    \centering
    \includegraphics[width=0.7\linewidth]{figures/eps/ECSIMTemporalConvergence_particles.eps}
    \caption{Error of position and velocity for the uncoupled particle mover compared to the analytical solution defined by CASE II for decreasing $\Delta t$.}
    \label{fig: ecsim-convergence-posvel-time}
\end{figure}

The field solver is dependent on the temporal and spatial discretisation, and thus, we show the convergence order for both. 
Figure \ref{fig: ecsim-convergence-elmag-time} shows the errors of the magnetic and electric field compared to the analytical solution for different time step sizes. Here we see that second-order accuracy is obtained in the time domain. Figure \ref{fig: ecsim-convergence-elmag-space} instead shows the error of these values with different grid cell sizes, indicating a second-order convergence on the spatial grid as well.
\begin{figure}[h]
\begin{subfigure}{0.49\linewidth}
  \includegraphics[width=\linewidth]{figures/eps/ECSIMTemporalConvergence_fields.eps}
  \subcaption{Error of the electric and magnetic field for the uncoupled field solver compared to the analytical solution defined by CASE II for decreasing $\Delta t$.}\label{fig: ecsim-convergence-elmag-time}  
\end{subfigure}
\hfill
\begin{subfigure}{0.49\linewidth}
  \includegraphics[width=\linewidth]{figures/eps/ECSIMSpatialConvergence.eps}
  \subcaption{Error of the electric and magnetic field for the uncoupled field solver compared to the analytical solution defined by CASE II for decreasing $\Delta x$.}\label{fig: ecsim-convergence-elmag-space}
\end{subfigure}
\end{figure}

We now perform the temporal analysis on the test cases CASE III and CASE IV as well. We use a spatial grid of $N_x = 512$ grid cells and simulate 10 000 particles. 
It is known that the coupling between the \textbf{particle mover} and \textbf{field solver} can break down the order of the two separate methods. The order of the coupled method depends on the number and size of the particles. This is why the results of CASE III and CASE IV will also be compared against the solution of the original paper of Lapenta \cite{lapenta_exactly_2017}. Figure \ref{fig: 1D-1V-convergence} shows the convergence of errors of the three state variables: position, velocity and electric field, for CASE III. The errors are computed against a reference solution using a time step size $50$ times smaller than the smallest size used in the plot. Only first-order convergence is observed. We do not show the magnetic field errors as the magnetic field does not interact with the particles in the 1D1V case due to the Lorentz force, which is used to calculate the acceleration of the particles
\[\textbf{F} = q\left(\textbf{E} + \textbf{v} \times \textbf{B}\right)\]
Since the cross-product of two vectors aligned along the same axis is always equal to $\textbf{0}$, this proves that the magnetic field does not influence the particles. Using the same reasoning on the curl equations in Maxwell's equations, we find that the magnetic field is constant, and the electric field dynamics are only influenced by the current. 
\begin{figure}[h]
    \centering
    \includegraphics[width=1\linewidth]{figures/eps/ECSIM1DConvergence.eps}
    \caption{Error of the state variables found by applying ECSIM to CASE III for decreasing $\Delta t$. The error is computed against a reference solution calculated using a times step size 50 times smaller than the smallest step size shown.}
    \label{fig: 1D-1V-convergence}
\end{figure}
% Table \ref{tab: 1D1V-convergence} shows the values of the convergence rate as a log reduction factor. This factor is calculated as $\frac{\log_{2}\left(\frac{E_i}{E_{i-1}}\right)}{\log_{2}\left(\frac{\Delta t_i}{\Delta t_{i-1}}\right)}$. A second-order convergence corresponds to $2$, while first-order accuracy is a $1$ in Table \ref{tab: 1D1V-convergence}.
% \begin{table}[h!]
% \centering
% \begin{tabular}{cccc}
% \toprule
% \textbf{Step size decrease} & \textbf{Position}& \textbf{Velocity} & \textbf{Electric Field}\\
% \midrule
% $0.01 \rightarrow 0.005$ & 1.28  & 0.89 & 1.98 \\
% $0.005\rightarrow0.0025$ & 1.98 & 1.81 & 1.98\\
% $0.0025\rightarrow0.00125$ & 1.93 & 2.01 & 1.96\\
% $0.00125\rightarrow0.000625$ & 1.78& 2.01&1.99\\
% \bottomrule
% \end{tabular}
% \caption{Convergence factor of ECSIM, calculated as $\frac{\log_{2}\left(\frac{E_i}{E_{i-1}}\right)}{\log_{2}\left(\frac{\Delta t_i}{\Delta t_{i-1}}\right)}$, on the 1D1V smooth problem showing the time step size and log reduction factor.}
% \label{tab: 1D1V-convergence}
% \end{table}

A 1D3V version is also analysed and shown in Figure \ref{fig: 1D-3V-convergence}. Only first-order accuracy is also observed for each variable. As previously stated these first-order accuracies could be due to the initial conditions or the number and shape of the particles as mentioned by Lapenta \cite{lapenta_exactly_2017}. Further comparisons must prove the correctness of the implementation.
\begin{figure}[h]
    \centering
    \includegraphics[width=1\linewidth]{figures/eps/ECSIM3DConvergence.eps}
    \caption{Error of the state variables found by applying ECSIM to CASE IV for decreasing $\Delta t$. The error is computed against a reference solution calculated using a times step size 50 times smaller than the smallest step size shown.}
    \label{fig: 1D-3V-convergence}
\end{figure}
% Table \ref{tab: 1D3V-convergence} shows the log reduction factors for the smooth 1D3V test case. As stated, the position only achieves first-order accuracy, while the other state variables are second-order accurate.
% \begin{table}[h!]
% \centering
% \begin{tabular}{ccccc}
% \toprule
% \textbf{Step size decrease} & \textbf{Position}& \textbf{Velocity} & \textbf{Electric Field} & \textbf{Magnetic Field}\\
% \midrule
% $0.01 \rightarrow 0.005$ & 1.43  & 1.96 & 1.65&1.83 \\
% $0.005\rightarrow0.0025$ & 1.11 & 2.11 & 1.87&1.96\\
% $0.0025\rightarrow0.00125$ & 1.02 & 2.03 & 1.96&1.99\\
% $0.00125\rightarrow0.000625$ & 1.01& 2.00&1.97&1.99\\
% \bottomrule
% \end{tabular}
% \caption{Convergence of ECSIM on the 1D3V smooth problem showing the time step size and log reduction factor.}
% \label{tab: 1D3V-convergence}
% \end{table}
 
 \subsubsection{Simulation Experiments}
 \label{sub: simulations}
We turn to simulation results for CASE III and CASE IV to further investigate the correctness of the code. We start with CASE III.

 Figure \ref{fig: 1D-1V-sim} shows the velocity distribution of the particles in 2D phase space for both the initial condition (top) and the state after $50$\,seconds (centre). The energy conservation (bottom) is also shown as a function of time for the two-stream instability. The phase space shows the creation of zones without electrons, as in the paper by Lapenta (2023) \cite{lapenta_advances_2023}. This is a known effect of the two-stream instability and, in combination with the energy conservation up to machine precision, leads us to accept the correctness of the 1D1V implementation.
\begin{figure}[h]
\centering
\begin{subfigure}{0.7\linewidth}
  \includegraphics[width=\linewidth]{figures/eps/Sim_plots_1D_init.eps}
  \subcaption{Phase space of CASE III at time $t=0$ s}\label{fig: ecsim-sim-1d1v-init}  
\end{subfigure}
\vfill
\begin{subfigure}{0.7\linewidth}
  \includegraphics[width=\linewidth]{figures/eps/Sim_plots_1D_end.eps}
  \subcaption{Phase space of CASE III at time $t=50$ s}\label{fig: ecsim-sim-1d1v-end}
\end{subfigure}
\vfill
\begin{subfigure}{0.7\linewidth}
  \includegraphics[width=\linewidth]{figures/eps/Sim_plots_1D_energy.eps}
  \subcaption{Error in energy of CASE III compared to the initial condition for each time step.}\label{fig: ecsim-sim-1d1v-energy}
\end{subfigure}
\caption{Serial simulation results of CASE III.}
\label{fig: 1D-1V-sim}
\end{figure}


% \begin{figure}[h]
%     \centering
%     \includegraphics[width=1\linewidth]{figures/eps/Sim_plots_1D.eps}
%     \caption{Phase space at the initial condition (top), after $50$ seconds (middle) and the error in energy (bottom) of the two-stream instability simulated using serial ECSIM.}
%     \label{fig: 1D-1V-sim}
% \end{figure}

For a test case with both magnetic and electric influence, we turn to CASE IV. While it is still 1-dimensional in space, it is three-dimensional in velocity and consequently keeps track of the electric and magnetic fields in three dimensions. Figure \ref{fig: 1D-3V-sim} shows the phase space and energy conservation of CASE IV. As in the paper by Lapenta (2023) \cite{lapenta_advances_2023}, islands are seen in the phase space, as well as an error in energy that hovers around machine precision. These observations allow us to assume the implementation is correct. On top of these tests, the results on precalculated initial conditions were also compared between the implemented \texttt{C++} code and the available \texttt{Matlab} code of Lapenta (2023) \cite{lapenta_advances_2023}.

\begin{figure}[h]
\centering
\begin{subfigure}{0.7\linewidth}
  \includegraphics[width=\linewidth]{figures/eps/Sim_plots_3D_init.eps}
  \subcaption{Phase space of CASE IV at time $t=0$\,s}\label{fig: ecsim-sim-1d3v-init}  
\end{subfigure}
\vfill
\begin{subfigure}{0.7\linewidth}
  \includegraphics[width=\linewidth]{figures/eps/Sim_plots_3D_end.eps}
  \subcaption{Phase space of CASE IV at time $t=62$\,s}\label{fig: ecsim-sim-1d3v-end}
\end{subfigure}
\vfill
\begin{subfigure}{0.7\linewidth}
  \includegraphics[width=\linewidth]{figures/eps/Sim_plots_3D_energy.eps}
  \subcaption{Error in the energy of CASE IV compared to the initial condition for each time step.}\label{fig: ecsim-sim-1d3v-energy}
\end{subfigure}
\caption{Serial simulation results of CASE IV.}
\label{fig: 1D-3V-sim}
\end{figure}

% \begin{figure}[h]
%     \centering
%     \includegraphics[width=1\linewidth]{figures/eps/Sim_plots_3D.eps}
%     \caption{Phase space at the initial condition (top), after $62$ seconds (middle) and the error in energy (bottom) of CASE IV simulated using serial ECSIM, blue: positive $\textbf{v}_y$, red: negative $\textbf{v}_y$.}
%     \label{fig: 1D-3V-sim}
% \end{figure}

We now use ECSIM for both the fine and coarse integrator in parareal to simulate CASE IV. The coarse solver uses a time step size of $10^{-2}$, while the step size of the fine solver is reduced further and further. A spatial discretisation of $N_x = 5000$ grid cells is used. This analysis gives the same convergence plot as the serial case as long as the tolerance used in parareal is more accurate than the lowest achieved accuracy for any of the state variables. Similar to CASE I, however, the required number of parareal iterations and evolution of the parareal errors are not encouraging. The error rises before going down, and the number of parareal iterations equals the number of time steps. As before, one can either coarsen the spatial grid discretisation or use smaller time step sizes for the coarse solver to improve this issue. This is likely again connected to accuracy issues due to a CFL-like condition occurring due to the instabilities. 
Figure \ref{fig: 1D-3V-parareal-convergence} shows the convergence of parareal for simulating CASE IV using 10 000 particles, a fine time step size of $10^{-4}$ and a coarse step of $10^{-2}$. 
\begin{figure}[h]
    \centering
    \includegraphics[width=0.6\linewidth]{figures/eps/3D_parareal_Convergence.eps}
    \caption{Parareal errors calculated during each parareal iteration, when applying parareal to CASE IV using a grid discretisation of 512 and 5000 cells. The coarse integrator uses a time step size of $\Delta t_\mathrm{Coarse} = 10^{-2}$, while the fine solver uses $\Delta t_\mathrm{Fine} = 10^{-4}$.}
    \label{fig: 1D-3V-parareal-convergence}
\end{figure}
These time step sizes limit the effective time range that is calculated. On top of this, one can wonder whether the results of parareal are trustworthy after more extended periods. Especially considering the test cases under investigation are instabilities showing highly non-linear behaviour. Parareal, however, has even been applied with success to chaotic systems \cite{d_samaddar_parallelization_2010}. For these chaotic systems, the individual solutions differed due to the predictor-corrector-like nature of the parareal algorithm. The statistical properties of these solutions, however, did get correctly simulated.

\subsubsection{Errors on State and Energy for Parareal}
\label{sub: errors}
It is worth noting that when parareal checks for convergence, it estimates the error compared to the fine solution as the relative change in the state variables over two consecutive iterations. This means that for a converging algorithm, the error compared to the fine solution should be bounded by the state change. This can be seen in Figure \ref{fig: error-vs-statechange} where the maximal error compared to the fine solution and the maximal parareal error are shown per iteration. Again, we note that the error compared to the serial fine solution is not too small compared to the calculated parareal error. This means the estimated error is a reasonably accurate approximation. As a result, the number of parareal iterations will not be impacted too much due to the use of this approximation.
\begin{figure}[h]
\centering
\begin{subfigure}{0.49\linewidth}
    \includegraphics[width=\linewidth]{figures/eps/est_vs_act_err_parareal_check.eps}
    \caption{Parareal error and the error compared to a serial fine solution during each parareal iteration of a simulation of CASE IV.}
    \label{fig: error-vs-statechange}
\end{subfigure}
\hfill
\begin{subfigure}{0.49\linewidth}
    \centering
    \includegraphics[width=\linewidth]{figures/eps/est_vs_energy_err_parareal_check.eps}
    \caption{Error in energy and the parareal error during each parareal iteration of a simulation of CASE IV.}
    \label{fig: energy-vs-statechange}
\end{subfigure}
\caption{Parareal error as upper bound for error compared to serial fine solution and energy error.}
\label{fig: parareal bound}
\end{figure}

% \begin{figure}[h]
%     \centering
%     \includegraphics[width=0.7\linewidth]{figures/eps/est_vs_act_err_parareal_check.eps}
%     \caption{Parareal error and the error compared to a serial fine solution during each parareal iteration of a simulation of CASE IV.}
%     \label{fig: error-vs-statechange}
% \end{figure}

While exact energy conservation is not a goal of the current work, it should be noted that for these test problems, it was not necessary to use a tolerance equal to the machine precision for parareal to obtain energy conservation up to machine precision. This manifests itself in the other experiments, where the error incurred in energy is significantly smaller than the estimated error for the state variables. This effect is shown in figure \ref{fig: energy-vs-statechange}, where the energy conservation of the solution is plotted for several iterations alongside the max state changes. The difference between the estimated error in the state variables and the error in the energy is multiple orders of magnitude, which indicates that the parareal error also gives a generous bound for the energy error.

These results are supplemented by the fact that for practical simulations, one often does not desire the fine solution to be accurate up to machine precision. This leads to the practical observation that one can often get very good (almost exact) energy conservation while using more lenient tolerances, e.g., $10^{-10}$ or $10^{-12}$. This is especially useful as it is often not desirable to approximate the fine solution up to machine precision since the fine solver itself is often only a discretised approximation of the actual solution.
%  \begin{figure}[h]
%     \centering
%     \includegraphics[width=0.7\linewidth]{figures/eps/est_vs_energy_err_parareal_check.eps}
%     \caption{Error in energy and the parareal error during each parareal iteration of a simulation of CASE IV.}
%     \label{fig: energy-vs-statechange}
% \end{figure}

\subsection{Tolerance of Linear Solver}
\label{sub: tol lin solver}
In section \ref{sec: linear solvers}, we investigate using different linear solvers to perform the field solver in ECSIM. The coarse integrator might benefit from using a different type of linear solver, e.g. direct or iterative, than the fine integrator. 
The use of an iterative solver, however, necessitates a user-specified tolerance to define convergence for the solution of the linear system. One of the findings of this work indicates that this choice cannot be made without considering the tolerance chosen for parareal. Our findings suggest that a tolerance must be chosen at least multiple orders of magnitude lower than the one used for parareal. Otherwise, the convergence of parareal is inhibited, and the error estimation during parareal is no longer trustworthy. This is demonstrated in Table \ref{tab: tolerance_lin_solver_8}, where the required amount of parareal iterations and the errors are shown for different ratios of tolerances while using GMRES. When the tolerances for parareal and the linear solver are both $10^{-8}$, the number of parareal iterations is much higher than when the tolerance for the linear solver is more stringent. The errors incurred compared to the serial solution also show concerning behaviour for a tolerance of $10^{-8}$ and $10^{-9}$. The parareal error is consistently lower than the error compared to the serial solution, invalidating its use as an upper bound. Only when a tolerance of $10^{-12}$ is chosen for the linear solvers does the parareal error become trustworthy again. Note that the error compared to the serial solution is already below the requested tolerance when $10^{-10}$ or $10^{-11}$ is the chosen tolerance for the serial solver. However, since the parareal error is higher than the error compared to the serial solution, this is not guaranteed as the calculated error is no longer an accurate estimate of the error.
\begin{table}[htbp]
    \centering
    \begin{tabular}{|c|c|c|c|c|c|}
        \hline
        \textbf{Tolerance}& $10^{-8}$ & $10^{-9}$ & $10^{-10}$ & $10^{-11}$& $10^{-12}$\\
        \hline
         Parareal iterations ($k$)& 6 & 2 & 2 & 2 & 2\\
         Parareal error & $4.0\cdot10^{-9}$ & $1.1\cdot10^{-9}$ & $2.0\cdot10^{-10}$ & $1.6\cdot10^{-10}$ & $1.6\cdot 10^{-10}$\\
         Error (w.r.t serial) & $2.8\cdot10^{-7}$ & $3.2\cdot10^{-8}$ & $4.4\cdot10^{-9}$ & $4.0\cdot10^{-10}$& $4.2\cdot 10^{-11}$\\
        \hline
    \end{tabular}
    \caption{Results of parareal on CASE I using different tolerances for the linear solver with a tolerance of $10^{-8}$ for parareal.}
    \label{tab: tolerance_lin_solver_8}
\end{table}
These results are confirmed in Table \ref{tab: tolerance_lin_solver_9} where the same tolerances are used for the linear solver, but the parareal tolerance is set to $10^{-9}$. The same conclusions can be drawn, being that the linear solver should use a tolerance several orders of magnitude more stringent than the parareal tolerance. A specific point of interest is the use of a tolerance of $10^{-8}$ for the linear solver while parareal has a tolerance of $10^{-9}$. In this case, the needed number of parareal iterations equals the number of time steps, but the error incurred by the parareal solution is lower than the requested tolerance. However, this situation is not useful in a speedup sense and we also assume that this error is not guaranteed due to the more lenient tolerance of the linear solver. The estimated error is also $0$, while the error compared to the serial solution is only $7.7\cdot10^{-10}$.
\begin{table}[htbp]
    \centering
    \begin{tabular}{|c|c|c|c|c|c|}
        \hline
        \textbf{Tolerance}& $10^{-8}$ & $10^{-9}$ & $10^{-10}$ & $10^{-11}$& $10^{-12}$\\
        \hline
         Parareal iterations ($k$) & 12 & 5 & 2 & 2 & 2\\
         Parareal error & $0.0$ & $6.5\cdot10^{-10}$ & $2.0\cdot10^{-10}$ & $1.6\cdot 10^{-10}$& $1.6\cdot 10^{-10}$\\
         Error (w.r.t serial) & $7.7\cdot 10^{-10}$ & $2.4\cdot 10^{-8}$ & $4.4\cdot 10^{-9}$ & $4.0\cdot 10^{-10}$ & $4.2\cdot 10^{-11}$\\
        \hline
    \end{tabular}
    \caption{Results of parareal on CASE I using different tolerances for the linear solver with a tolerance of $10^{-9}$ for parareal.}
    \label{tab: tolerance_lin_solver_9}
\end{table}


\section{Parallel Performance}
In the previous section, we investigated whether our serial implementations are correct and when the parareal algorithm gives efficient and trustworthy results. We will now perform parameter tests to obtain more insight into the parallel performance of parareal. We attempt to find the best coarse and fine propagators to obtain the largest speedup and smallest computational runtime. We perform all our experiments on CASE IV.
\subsection{Temporal Coarsening}
\label{sec: temporal coarsening}
The choice of fine and coarse integrators for parareal can massively influence the algorithm's performance. It is thus crucial to decide on a good combination. In this subsection, we consider reducing complexity by changing parameters related to the time domain.
\subsubsection{Time step size}
\label{sub: step size}
 It is well-known that using smaller time step sizes in ODE solvers results in more accurate solutions than larger sizes. Using the same integrator method for both the coarse and fine solver while employing a large time step size for the coarse solver and a smaller step size for the fine solver thus constitutes a natural choice for a parareal implementation. The coarse solver is expected to yield a ``rougher'' solution estimate at every coarse time point compared to the fine solver. Suppose $r$ fine time steps are computed for each coarse time step using an ODE solver of order $l$. In this case, the error incurred by the fine integrator is expected to be a factor $\mathcal{O}(r^l)$ smaller than the coarse approximation. 
This choice of fine and coarse solver also allows for easy comparisons in terms of computational complexity between the fine and coarse. If the fine solver performs $100$ steps for each coarse step, one would expect that the fine integrator is $100$ times slower than the coarse solver.

To define the fine and coarse time step sizes, we fix one and calculate the other based on the chosen ratio $r$. We now show how this factor influences the performance of parareal by plotting the speedup for increasing $r$.

The first experiment chooses the coarse step size as a constant and reduces the fine size. Although this allows us to show an ever-increasing speedup, it is practically less valuable. In a typical use case, a certain accuracy is desired over the simulation time period. Since parareal will approach the accuracy of the fine solver, the fine integrator should reach the desired accuracy. This means the fine time step size is connected to the desired accuracy, and as such, it does not make sense to go to much smaller step sizes, even if the speedup would be better. Decreasing the fine time step size still increases the computational runtime of the parareal algorithm, the higher speedup only indicates that it rises less steeply than the runtime of the serial solution. Therefore, it is also informative to show how the speedup and computational runtimes change when the fine time step size is fixed.

Figure \ref{fig: temporal-coarsening} shows the computational runtime (left) and speedup (right) obtained when choosing either a fixed coarse time step size (top) or fine step size (bottom). All experiments use 96 cores of one node on the VSC and have 512 grid cells for spatial discretisation. The top graphs use a fixed coarse time step size, $\Delta t_\mathrm{Coarse} = 10^{-3}$, while the bottom plots use a constant fine step size $\Delta t_\mathrm{Fine} = 10^{-5}$. 

In Figure \ref{fig: constant coarse time}, as stated before, the computational runtime increases as the factor between fine and coarse time step sizes increases. Confirming that needlessly increasing the accuracy of the fine solver is not advised. Note that the scales on the serial and parareal solver axes differ, where the serial solve is more expensive than the parareal solution. 
Figure \ref{fig: constant coarse single speedup} shows that making the fine solver more expensive helps attain higher speedup. This means it is beneficial to perform parareal when a very fine discretisation is desired. This upward motion is only warranted by the assumption that the number of needed parareal iterations remains the same, which is always $3$ in the tested results (except when $\frac{\Delta t_\mathrm{Coarse}}{\Delta t_\mathrm{Fine}}=1$). This is shown by the theoretical bound, which approximates the speedup equation \ref{eq: speedup} by neglecting the parallel communication. The required number of iterations likely remains the same due to the coarse timestep already being very small to account for the accuracy constraints mentioned before. This allows parareal to converge quickly. Note that the speedup graph seems to follow the general outline of the theoretical bound. It is surprising, however, that the difference is larger for smaller $\Delta t_\mathrm{Fine}$ as we expect the parallel overhead to be amortized because of the expensive fine integrator. A speedup of $18.7$ is achieved for a factor of $512$.

The bottom graphs of Figure \ref{fig: temporal-coarsening} show the results of keeping the fine time step size constant while increasing the step size of the coarse integrator. To obtain comparable results for the speedup of the test with constant fine solver step size, the time frame was always chosen to be equal to $96\cdot\Delta t_\mathrm{Coarse}$. This explains why the computational runtime of the serial solver increases in Figure \ref{fig: constant fine time}. This measure ensures the parallel section can always appoint exactly one core per coarse time step, putting more emphasis on the parallel aspect of the parareal algorithm. In Figure \ref{fig: constant fine single speedup}, it can be seen that, as the size of the coarse time steps increases, the speedup obtained increases to a certain point before it plummets. This may be explained as follows: choosing a very cheap solver reduces serial computational time, thereby speeding up the simulations. However, the cheap coarse solver may not yield accurate solutions, necessitating additional parareal iterations that can negate the computational gains obtained from choosing a cheap coarse solve. The theoretical bound again follows the behaviour of the experimental data closely, where the difference is again the largest at the point with the largest speedup. We obtain the largest speedup of $12.4$ in this scenario for a ratio $\frac{\Delta t_\mathrm{Coarse}}{\Delta t_\mathrm{Fine}}= 128$. 
For this ratio, the approximation of the coarse solver is accurate enough for the parareal algorithm to converge within a reasonable number of iterations without incurring significant overhead. Further reductions in the coarse time step size seem to be counter-productive. Although they decrease the number of iterations, the added cost is too large.

\begin{figure}[h]
\begin{subfigure}{0.49\linewidth}
\centering
  \includegraphics[width=\linewidth]{figures/eps/time_step_constant_coarse_time.eps}
  \subcaption{Computational runtime of serial fine solution and parareal solution for different fine time step sizes. $\Delta t_\mathrm{Coarse} = 10^{-3}$}\label{fig: constant coarse time}  
\end{subfigure}
\hfill
\begin{subfigure}{0.49\linewidth}
\centering
  \includegraphics[width=\linewidth]{figures/eps/time_step_constant_coarse_single_speedup.eps}
  \subcaption{Speedup of parareal and the theoretical speedup neglecting parallel overhead for different fine time step sizes. $\Delta t_\mathrm{Coarse} = 10^{-3}$}\label{fig: constant coarse single speedup}
\end{subfigure}
 \vskip\baselineskip
\begin{subfigure}{0.49\linewidth}
\centering
  \includegraphics[width=\linewidth]{figures/eps/time_step_constant_fine_time.eps}
  \subcaption{Computational runtime of serial fine solution and parareal solution for different coarse time step sizes. $\Delta t_\mathrm{Fine} = 10^{-5}$}\label{fig: constant fine time}
\end{subfigure}
\hfill
\begin{subfigure}{0.49\linewidth}
\centering
  \includegraphics[width=\linewidth]{figures/eps/time_step_constant_fine_single_speedup.eps}
  \subcaption{Speedup of parareal and the theoretical speedup neglecting parallel overhead for different coarse time step sizes. $\Delta t_\mathrm{Fine} = 10^{-5}$}\label{fig: constant fine single speedup}
\end{subfigure}
\caption{Temporal coarsening experiments simulating CASE IV.}
\label{fig: temporal-coarsening}
\end{figure}

For the fixed $\Delta t_\mathrm{Coarse}$, we expect that the increase of the speedup along with the factor $\frac{\Delta t_\mathrm{Coarse}}{\Delta t_\mathrm{Fine}}$ also occurs for different starting $\Delta t_\mathrm{Coarse}$. This behaviour is shown in Figure \ref{fig: temporal-coarsening-speedup_coarse_const}. The increase can be seen for all cases, although for $\Delta t_\mathrm{Coarse} = 10^{-2}$, it is damped. This case is quite close to the accuracy constraint observed in the previous section and thus has a large amount of parareal iterations where the parareal error is not properly converging.
 \begin{figure}[h]
    \centering
    \includegraphics[width=0.7\linewidth]{figures/eps/time_step_constant_coarse_speedup.eps}
    \caption{Speedup of parareal simulations of CASE IV with temporal coarsening using decreasing $\Delta t_\mathrm{Fine}$, starting from different constant $\Delta t_\mathrm{Coarse}$.}
    \label{fig: temporal-coarsening-speedup_coarse_const}
\end{figure}
We also expect to see the peak shown in Figure \ref{fig: constant fine single speedup} for different $\Delta t_\mathrm{Fine}$. This is shown in Figure \ref{fig: temporal-coarsening-speedup_fine_const}. The peak for $\Delta t_\mathrm{Fine} =10^{-4}$ occurs at $\frac{\Delta t_\mathrm{Coarse}}{\Delta t_\mathrm{Fine}} = 32$, while the highest speedup for $\Delta t_\mathrm{Fine} = 10^{-6}$ is expected to appear at a ratio beyond $512$. The reason why the turning point moves to higher ratios $\frac{\Delta t_\mathrm{Coarse}}{\Delta t_\mathrm{Fine}}$ for smaller $\Delta t_\mathrm{Fine}$ is because the fine time step size is smaller. Thus, the ratio of the fine and coarse sizes can be larger before the coarse integrator becomes too inaccurate.
 \begin{figure}[h]
    \centering
    \includegraphics[width=0.7\linewidth]{figures/eps/time_step_constant_fine_speedup.eps}
    \caption{Speedup for parareal with temporal coarsening using different constant $\Delta t_\mathrm{Fine}$, simulating CASE IV.}
    \label{fig: temporal-coarsening-speedup_fine_const}
\end{figure}

\subsubsection{Subcycling}
\label{sub: subcycling}

The standard PIC algorithm consists of four steps, performed cyclically. Firstly, the particles move during the \textbf{particle mover}, after which they are collected and projected onto the grid to compute the current and charge density. These currents and charge densities at each grid cell are then used during the \textbf{field solver} section to calculate the change in electric and magnetic fields. These new forces are then projected onto the particles, whose movement will be influenced in the next \textbf{particle mover} step. If we reduce or increase the time step size, this does not influence this cycle. Subcycling, however, does change the underlying iteration scheme.

We now turn our attention to using subcycling for the coarse solver in parareal. Subcycling can be used during the coarse integrator to obtain a more accurate approximation of the fine solution without incurring the high costs of reducing the coarse time step size, which would require also computing the field solver. A second approximation is made during the implemented subcycled ECSIM code; the velocity is kept constant across the subcycles \cite{lapenta_advances_2023}. This allows us to calculate all of the subcycles in parallel. This parallelisation can use the cores that are not used during the serial calculations of parareal. Since each subcycle can be calculated in (almost) perfect parallelism, the expected cost of subcycling is negligible under the constraint that the number of subcycles remains lower than or equal to the number of cores available. Our ECSIM code implements subcycling by injecting $\nu-1$ extra equispaced time points in each time step, at which only the position is updated. These positions are then averaged during the \textbf{field solver} to obtain a more accurate solution. This hopefully decreases the number of iterations parareal needs to converge. The fine solver uses a time step size of $10^{-5}$, while the coarse step size is equal to $10^{-3}$.

It can be seen in Figure \ref{fig: temporal-subcycling-errors} that the accuracy does indeed increase for more subcycles. For example, the version with $10$ subcycles has a parareal error of $2.8\cdot 10^{-8}$ at iteration $9$, while the simulation without subcycles has an error of $5.4\cdot 10^{-8}$. The increased accuracy can also be noticed in the number of time steps that converge after each parareal iteration, which increase along with the number of subcycles. Figure \ref{fig: temporal-subcycling}, however, shows that one fails to obtain any improvement in the computational cost incurred. This is due to parallelisation overhead and the need to average the positions of the particles after the subcycles. We note that the difference in how many time steps have converged at an iteration can be quite different between the subcycled and unsubcycled versions. For example, for the unsubcycled version, 16 steps have converged after the second iteration, while the version with $\nu = 10$ already has 75 converged time steps. The decreased parareal errors and increased number of time steps converging per parareal iteration show that increased performance might be attainable in certain scenarios. Example situations would involve many time steps and exhibit relatively poor convergence for the parareal algorithm, requiring numerous iterations.

  \begin{figure}[h]
    \centering
    \includegraphics[width=0.6\linewidth]{figures/eps/subcycle_speedup.eps}
    \caption{Speedup of the parareal simulation of CASE IV compared to a serial fine solve, using different amounts of subcycling in the coarse integrator.}
    \label{fig: temporal-subcycling}
\end{figure}

  \begin{figure}[h]
    \centering
    \includegraphics[width=0.6\linewidth]{figures/eps/subcycle_parareal_convergence.eps}
    \caption{Parareal errors during each parareal iteration for different amounts of subcycling in the coarse integrator, simulating CASE IV. }
    \label{fig: temporal-subcycling-errors}
\end{figure}

\subsection{Parallel Scaling}
\label{sec: parallel scaling}
% In chapter \ref{cha: pint}, two strategies are discussed to simulate long time frames; [A] each processor gets multiple coarse steps assigned to it, or [B] the time domain is split into separate chunks on which parareal is performed in serial. These can be used when the number of available cores is insufficient to assign a separate processor to each coarse time step. 
We now discuss the scalability with respect to the number of cores used. This information is crucial in deciding which strategy is most fitting for the desired simulation time period. This section assumes that the number of time steps equals the number of cores. Figure \ref{fig: core scaling speedup} shows how the speedup changes depending on the number of time steps, while \ref{fig: core scaling efficiency} shows the parallel efficiency in the same circumstances. The speedup steadily rises with the increase in cores, however, the parallel efficiency shows that this increase is not proportionate with the extra cores used. It insinuates that it can be more efficient to split up the time domain into multiple sequential parareal sections when long-duration simulations are desired. Note that this will increase the computational runtime of the algorithm and should only be considered in the context of efficient usage of cores and the associated energy costs. 
To remedy the decreasing returns, one could consider larger time step sizes for the coarse solver to more quickly cover large time spans. Special care should be taken, that the convergence of parareal is not impeded due to accuracy constraints in the coarse solver. The decrease in parallel efficiency might necessitate the use of multiple computing nodes using \texttt{MPI} or offloading to GPU. This would constitute a subject for future work. Again, the theoretical bounds from equations \ref{eq: speedup} and \ref{eq: parallel efficiency} are shown, where the parallel communication cost is neglected. Keeping this in mind, we can see that the number of iterations needed for larger time frames only increases once from $2$ to $3$, shown by a sharp decrease in theoretical parallel efficiency at $18$ time steps. The experimental results seem to follow the general dynamics of the theoretical bound.

\begin{figure}[h]
\centering
\begin{subfigure}{0.49\linewidth}
    \includegraphics[width=\linewidth]{figures/eps/core_scaling_test_speedup.eps}
    \caption{Speedup}
    \label{fig: core scaling speedup}
\end{subfigure}
\hfill
\begin{subfigure}{0.49\linewidth}
    \centering
    \includegraphics[width=\linewidth]{figures/eps/core_scaling_test_efficiency.eps}
    \caption{Parallel efficiency}
    \label{fig: core scaling efficiency}
\end{subfigure}
\caption{Parallel performance of the parareal algorithm applied to CASE IV for an increasing number of cores, where each core is assigned to one coarse time step, $N_x = 512$, $\Delta t_\mathrm{Coarse} = 10^{-3}$, $\Delta t_\mathrm{Fine} = 10^{-5}$.}
\label{fig: core scaling}
\end{figure}


\subsection{Linear Solvers}
\label{sec: linear solvers}
Here, we investigate the performance of some widely used (direct and indirect) solvers to approximate the solution to the linear system of equations obtained by discretising Maxwell's equations. 
% Direct solvers have the benefit of computing the solution \textit{exactly}; however, they may be computationally prohibitive for large systems. In a serial context, the solution should be accurate up to machine precision due to the desired exact energy conservation. Since the basic parareal algorithm does not conserve this property \cite{gander_analysis_2014}, it would have to iterate to highly stringent tolerances to achieve machine precision accuracy. Because total energy conservation is not a requirement for this work, more options are available to solve this linear system. The matrix resulting from the linear system is sparse and has a block-diagonal structure, which lends itself to iterative solvers, such as the generalised minimal residual method (GMRES). Unfortunately, the discretisation matrix is not symmetric, invalidating specific solvers, such as Cholesky-based decompositions. 
We consider the LU decomposition method along with the commonly used iterative solvers, GMRES and BiCGSTAB outlined in Chapter \ref{cha: methodology}. We investigate the effect of choosing different combinations of these three, for the coarse and fine solvers. It is known that the computational cost of an iterative solver depends on the time step size \cite{einkemmer_adaptive_2018, hochbruck_exponential_1998}, whereas the cost of a direct solver is independent of it. We also test different preconditioners to speed up the convergence of the iterative solvers.

\subsubsection{Combinations of Linear Solvers}
\label{sec: comb lin solv}
We first consider the use of different linear solvers. Specific combinations of direct and iterative solvers could allow for better speedup or computational runtime. A possible strategy is to use a very accurate but slow direct LU solver in the fine solution for high accuracy while using a less accurate iterative solver in the coarse. However, a coarse timestep could lead to more iterations needed for the iterative solvers to converge to the desired accuracy, thus increasing the overall computational time. Figure \ref{fig: linear solver combination} shows the results of the same simulation using each linear solver combination. The left panel shows the speedup achieved, and the right panel illustrates the computational runtime for each combination under consideration. Each cluster of three bars indicates the use of a different fine solver. Each blue bar indicates a simulation where the coarse solver uses an LU solver, the orange bars represent GMRES coarse solvers, and the green corresponds to coarse BiCGSTAB. While both bar plots are important, the right plot is essential in understanding the influence of the different solvers. The bar plot on the left clearly shows a preference for a costly LU solver for the fine integrator. However, this may be misleading as one has to consider the overall computational runtime. The figure on the right shows that it is best to use iterative solvers for both the fine and coarse solvers, and there is no significant difference between GMRES and BiCGSTAB. This discrepancy in interpretation is caused by the difference in computational runtime caused by the multiple fine solvers. The preference for iterative solvers for both coarse and fine propagators is most likely caused by the reasonably small coarse time step size ($\Delta t_\mathrm{Coarse} = 10^{-3}$). This choice was made to ensure proper parareal convergence. 
  \begin{figure}[h]
    \centering
    \includegraphics[width=1\linewidth]{figures/eps/solver_test.eps}
    \caption{Left: Speedup for parareal simulations of CASE IV using different combinations of linear solvers for coarse and fine. Right: Time needed for parareal simulations of CASE IV using different combinations of linear solvers for coarse and fine.}
    \label{fig: linear solver combination}
\end{figure}

As expected of parareal, the solutions for any combination of linear solvers are all equally accurate to their respective serial solutions. 

\subsubsection{Preconditioners}
\label{sec: precond}
We now investigate the influence of different preconditioners on the computational runtime of the linear solver. 
Preconditioners are transformations performed on the linear system to improve the convergence rate of linear solvers. Here, we define preconditioners to be the inverse of a matrix $\textbf{P}$ defined so that $\textbf{P}^{-1}\textbf{A}$ has an increased convergence rate for iterative methods that solve $\textbf{Ax} = \textbf{b}$. This means a preconditioned system will solve $\textbf{P}^{-1}\textbf{Ax} = \textbf{P}^{-1}\textbf{b}$. Three different types of preconditioners will be tested. The first preconditioner is the identity matrix, i.e., no preconditioning. The second is the Jacobi preconditioner, for which the preconditioning matrix corresponds to the diagonal of the matrix $\textbf{A}$; it should perform well for diagonally dominant matrices. The final preconditioner that is examined is the incomplete LU factorisation. The general idea of incomplete LU is to factorise $\textbf{A}$ into the product of a lower and upper triangular matrix such that $\textbf{A} \approx \textbf{LU}$, where $\textbf{L}$ and $\textbf{U}$ share a sparsity pattern with $\textbf{A}$. This reduces memory and computational cost requirements whilst providing a rough estimate of the solution. One can then use this solution as a starting point.

Figure \ref{fig: preconditioners} shows the performance of the three preconditioners under consideration for different time step sizes for a serial implementation of CASE IV. While there is no significant difference between the identity and Jacobi preconditioners, one can see that the incomplete LU decomposition is faster when the time step size is larger and slower than the other preconditioners when the step size is reduced. While the incomplete LU preconditioner is more expensive to calculate, the iterative solver only needs a small number of iterations afterwards to solve the system for the considered test. This contrasts the identity and Jacobi preconditioners, which are fast to perform but might not help the solver converge fast enough, leading to many iterations.
 \begin{figure}[h]
    \centering
    \includegraphics[width=1\linewidth]{figures/eps/precond_test.eps}
    \caption{Time needed for serial solve of CASE IV using different preconditioners for GMRES.}
    \label{fig: preconditioners}
\end{figure}

\section{Discussion}
We close this chapter with a discussion of our results. The convergence of parareal must be discussed. It is clear that the convergence of parareal strongly depends on the accuracy of the coarse propagator. Parareal requires an accurate coarse solver, or will not converge in an orderly fashion. As a result, one would prefer to use parareal on smooth problems as seen in Section \ref{sub: convergence parareal}. To resolve problems with highly oscillatory behaviour, the coarse solver must be selected such that the ratio $\frac{\Delta t_\mathrm{Coarse}}{\Delta x_\mathrm{Coarse}}$ is small enough. Our implementation only allows for coarsening in the temporal domain ($\Delta x_\mathrm{Coarse} = \Delta x_\mathrm{Fine}$). Thus, the grid cell size is, in practice, often fixed due to the desired accuracy. This means that the time step size must be reduced to ensure the correct accuracy. This, however, heavily influences the possible coarse and fine propagators. For one, the small coarse step size gives preference to iterative solvers for the coarse solver. Due to the small difference between time steps, quicker and cheaper preconditioners are also better. The required small timesteps for the coarse solver also mean that many steps would be required to cover more extended periods. Section \ref{sec: parallel scaling} shows that this reduces the parallel efficiency of the algorithm. The sudden increase in parareal iterations when the ratio $\frac{\Delta t_\mathrm{Coarse}}{\Delta x_\mathrm{Coarse}}$ is too large, also shows itself in a very well pronounced best coarse time step size for any given fine time step size. Increasing the coarse step size more than this heavily decreases the obtainable speedup. The coarse time step size also sets a minimum for the fine time step size. The best use cases are thus found where a very high accuracy in the time domain is desired.

Better performance could be achieved if the ratio $\frac{\Delta t_\mathrm{Coarse}}{\Delta x_\mathrm{Coarse}}$ could be decoupled from the fine solution. In this scenario, the coarse propagator could be defined with a larger grid cell size than the fine solver, allowing for larger coarse time step sizes. This could be accomplished by coarsening in the spatial domain and is, therefore, an interesting direction for further studies.


%%% Local Variables: 
%%% mode: latex
%%% TeX-master: "thesis"
%%% End:
% \chapter{Parareal Solvers for Particle-in-Cell Simulations: Parallel Performance}
\label{cha: speedup}

%%% ============================================================================================ %%% 
In this chapter, we use the test cases and methods described in Chapter \ref{cha: methodology} to investigate the computational runtime and speedup for different parameters by changing the coarse and fine propagators. 
 The following parameter studies are considered: 
 \begin{itemize}
 
    \item Temporal coarsening
    
    \item Scaling tests on shared-memory architecture
    
    \item Linear solvers (used during ECSIM)
 
 \end{itemize}
A spatial coarsening test was also considered (Appendix \ref{app: spatial coarsening}). However, elementary tests revealed that studying the efficacy of spatial coarsening would require a comprehensive analysis with various high-order polynomial interpolation methods, which is beyond this project's scope. 

\section{Temporal Coarsening}
The choice of fine and coarse integrators for parareal can massively influence the algorithm's performance. It is thus crucial to decide on a good combination. In this section, we consider reducing complexity by changing parameters related to the time domain.
\subsection{Time step size}
 It is well-known that using smaller time step sizes in ODE solvers results in more accurate solutions than larger sizes. Using the same integrator method for both the coarse and fine solver while employing a large time step size for the coarse solver and a smaller step size for the fine solver thus constitutes a natural choice for a parareal implementation. The coarse solver is expected to yield a ``rougher'' solution estimate at every coarse time point compared to the fine solver. Suppose $r$ fine time steps are computed for each coarse time step, using an ODE solver of order $l$. In this case, the error incurred by the fine integrator is expected to be a factor $\mathcal{O}(n^r)$ smaller than the coarse approximation. 
This choice of fine and coarse solver also allows for easy comparisons in terms of computational complexity between the fine and coarse. If the fine solver performs $100$ steps for each coarse step, one would expect that the fine integrator is $100$ times slower than the coarse solver.

To define the fine and coarse time step sizes, we fix one and calculate the other based on the chosen ratio $r$. We now show how this factor influences the performance of parareal by plotting the speedup for increasing $r$.

The first experiment chooses the coarse step size as a constant and reduces the fine size. Although this allows us to show speedup results approaching the theoretical limit from \ref{eq: speedup}, it is practically less valuable. In a typical use case, a certain accuracy is desired over the simulation time period. Since parareal will approach the accuracy of the fine solver, the fine integrator should reach the desired accuracy. This means the fine time step size is connected to the desired accuracy, and as such, it does not make sense to go to much smaller step sizes, even if the speedup would be better. Decreasing the fine time step size still increases the computational runtime of the parareal algorithm, the higher speedup only indicates that it rises less steeply than the runtime of the serial solution. Therefore, it is also informative to show how the speedup and computational runtimes change when the fine time step size is fixed. This plot can be used to choose the best coarse integrator step size for a given fine step size.

Figure \ref{fig: temporal-coarsening} shows the computational runtime (left) and speedup (right) obtained when choosing either a fixed coarse time step size (top) or fine step size (bottom). All experiments use 96 cores of one node on the VSC and have 512 grid cells for spatial discretisation. The top graphs use a fixed coarse time step size, $\Delta t_\mathrm{Coarse} = 10^{-3}$, while the bottom plots use a constant fine step size $\Delta t_\mathrm{Fine} = 10^{-5}$. 

In the top plots, as stated before, the computational runtime increases as the factor between fine and coarse time step sizes increases. Confirming that needlessly increasing the accuracy of the fine solver is not advised. Note that the scales on the serial and parareal solver axes differ, where the serial solve is more expensive than the parareal solution. 
The top right graph shows that making the fine solver more expensive helps attain higher speedup. This means it is beneficial to perform parareal when a very fine discretisation is desired. This upward motion is only warranted by the assumption that the number of needed parareal iterations remains the same, which is always $3$ in the tested results (except when $\frac{\Delta t_\mathrm{Coarse}}{\Delta t_\mathrm{Fine}}=1$). This is shown by the theoretical bound shown in the figure, which approximates the speedup equation \ref{eq: speedup} with the ratio between the used number of cores to the number of required parareal iterations, $\frac{p}{k} = 48$. The required number of iterations likely remains the same due to the coarse timestep already being very small to account for the CFL conditions of the ECSIM algorithm. This allows parareal to converge quickly. Note that the speedup graph seems to approach the theoretical bound as $\frac{\Delta t_\mathrm{Coarse}}{\Delta t_\mathrm{Fine}}$ increases, validating the conclusions of equation \ref{eq: speedup}. A speedup of $27$ is achieved for a factor of $512$.

The bottom graphs of Figure \ref{fig: temporal-coarsening} show the results of keeping the fine time step size constant while increasing the step size of the coarse integrator. To obtain comparable results for the speedup of the test with constant fine solver step size, the time frame was always chosen to be equal to $96\cdot\Delta t_\mathrm{Coarse}$. This explains why the computational runtime of the serial solver increases in the bottom left plot. This measure ensures the parallel section can always appoint exactly one core per coarse time step, putting more emphasis on the parallel aspect of the parareal algorithm. In the bottom right figure, it can be seen that, as the number of coarse time steps increases, the speedup obtained increases to a certain point before it plummets. This may be explained as follows: choosing a very cheap solver reduces serial computational time, thereby speeding up the simulations. However, the cheap coarse solver may not yield reasonably accurate solutions, necessitating additional parareal iterations that can negate the computational gains obtained from choosing a cheap coarse solve. The theoretical bound shows that the number of iterations needed by parareal decreases as the coarse and fine time step sizes get closer. We obtain the largest speedup in this scenario for a $\Delta t_G/\Delta t_F = 128$. 
Here, the approximation of the coarse solver is accurate enough for the parareal algorithm to converge within a reasonable number of iterations without incurring significant overhead. Further reductions in the coarse time step size seem to be counter-productive. Although they decrease the number of iterations, the added cost is too large.

 \begin{figure}[h]
    \centering
    \includegraphics[width=1\linewidth]{figures/png/time_step_constant_coarse.png}
    \includegraphics[width=1\linewidth]{figures/png/time_step_constant_fine.png}
    \caption{Computational runtime and speedup of parareal using temporal coarsening during the simulation of the transverse stream instability.}
    \label{fig: temporal-coarsening}
\end{figure}
For the fixed $\Delta t_\mathrm{Coarse}$, we expect that the increase of the speedup along with the factor $\frac{\Delta t_\mathrm{Coarse}}{\Delta t_\mathrm{Fine}}$ also occurs for different starting $\Delta t_\mathrm{Coarse}$. This behaviour is shown in Figure \ref{fig: temporal-coarsening-speedup_coarse_const}. The increase can be seen for all cases, although for $\Delta t_\mathrm{Coarse} = 10^{-2}$, it is damped. This case is quite close to the CFL condition of the PIC method and thus has a large amount of parareal iterations where the parareal error is not properly converging. As stated before, the time periods which are simulated are dependent on the coarse time step size, so one should not use this plot to choose between $\Delta t_\mathrm{Coarse}$ when a specific time range is desired.
 \begin{figure}[h]
    \centering
    \includegraphics[width=0.7\linewidth]{figures/png/time_step_constant_coarse_speedup.png}
    \caption{Speedup of parareal with temporal coarsening using different constant $\Delta t_\mathrm{Coarse}$.}
    \label{fig: temporal-coarsening-speedup_coarse_const}
\end{figure}
We also expect to see the peak for different $\Delta t_\mathrm{Fine}$. This is shown in Figure \ref{fig: temporal-coarsening-speedup_fine_const}. The peak for $\Delta t_\mathrm{Fine} =10^{-4}$ occurs at $\frac{\Delta t_\mathrm{Coarse}}{\Delta t_\mathrm{Fine}} = 32$, while the highest speedup for $\Delta t_\mathrm{Fine} = 10^{-6}$ is expected to appear at a ratio beyond $512$. The reason why the turning point moves further along the x-axis is because the fine time step size is smaller. Thus, the ratio of the fine and coarse sizes can be larger before the coarse integrator becomes too inaccurate.
 \begin{figure}[h]
    \centering
    \includegraphics[width=0.7\linewidth]{figures/png/time_step_constant_fine_speedup.png}
    \caption{Speedup for parareal with temporal coarsening using different constant $\Delta t_\mathrm{Fine}$.}
    \label{fig: temporal-coarsening-speedup_fine_const}
\end{figure}

\subsection{Subcycling}
\label{sub: subcycling}

The standard PIC algorithm consists of four steps, performed cyclically. Firstly, the particles move during the \textbf{particle mover}, after which they are collected and projected onto the grid to compute the current and charge density. These currents and charge densities at each grid cell are then used during the \textbf{field solver} section to calculate the change in electric and magnetic fields. These new forces are then projected onto the particles, whose movement will be influenced in the next \textbf{particle mover} step. If we reduce or increase the time step size, this does not influence this cycle. There is, however, an adaptation of the algorithm that does change the underlying iteration scheme. The modification in question is called subcycling and consists of performing the \textbf{particle mover} multiple times before moving on to the \textbf{field solver}. In this setting, the time step is subdivided into $\nu$ (not necessarily equal sized) substeps, $\Delta t_\nu$. The particles then move $\nu$ times influenced by the same constant fields, after which the (weighted) average of the positions and velocities are used to update the fields. These fields are then projected onto the particles, and the cycle starts anew. Subcycling can reduce computational costs in situations where the dynamics of the particles are much faster than those of the fields. This allows for the time step size, $\Delta t$, to be chosen based on the dynamics of the slower fields. In contrast, the substep sizes $\Delta t_\nu$ are chosen to be small enough to accurately approximate the movement of the particles. Often, particles move in cyclotron orbits; in situations like these, it is also possible to use subcycling for gyro-averaging. This is done by selecting $\Delta t$ to step over the gyration time scale while using the $\Delta t_\nu$ to average the movement during the gyromotion.

We now turn our attention to using subcycling for the coarse solver in parareal. Subcycling can be used during the coarse integrator to obtain a more accurate approximation of the fine solution without incurring the high costs of reducing the coarse time step size, which would require performing more steps. A second approximation is made during the implemented subcycled ECSIM code; the velocity is kept constant across the subcycles \cite{lapenta_advances_2023}. This allows us to calculate all of the subcycles in parallel. This parallelisation can use the cores that are not used during the serial calculations of parareal. Since each subcycle can be calculated in (almost) perfect parallelism, the expected cost of subcycling is negligible under the constraint that the number of subcycles remains lower than or equal to the number of cores available. Our ECSIM code implements subcycling by injecting $\nu-1$ extra equispaced time points in each time step, at which only the position is updated. These positions are then averaged during the \textbf{field solver} to obtain a more accurate solution. This hopefully decreases the number of iterations parareal needs to converge. The fine solver uses a time step size of $10^{-5}$, while the coarse step size is equal to $10^{-3}$.

It can be seen in Figure \ref{fig: temporal-subcycling-errors} that the accuracy does indeed increase for more subcycles. For example the version with $10$ subcycles has a parareal error of $2.8\cdot 10^{-8}$ at iteration $9$, while the simulation without subcycles has an error of $5.4\cdot 10^{-8}$.  The increased accuracy can also be noticed in the number of time steps that converge after each parareal iteration, which increase along with the number of subcycles. Figure \ref{fig: temporal-subcycling}, however, shows that one fails to obtain any improvement in the computational cost incurred. This is due to parallelisation overhead and the need to average the positions of the particles after the subcycles. These experiments converge quickly, only needing $3$ iterations to converge. The difference in how many time steps have converged at an iteration can be quite different, however. For example, for the unsubcycled version, 16 steps have converged after the second iteration, while the version with $\nu = 10$ already has 75 converged time steps. The decreased parareal errors and increased number of time steps converging per parareal iteration show that increased performance might be attainable in certain scenarios. Example situations would involve many time steps and exhibit relatively poor convergence for the parareal algorithm, requiring numerous iterations.

  \begin{figure}[h]
    \centering
    \includegraphics[width=0.7\linewidth]{figures/png/subcycle_speedup.png}
    \caption{Speedup of the parareal simulation compared to a serial fine solve, using different amounts of subcycling.}
    \label{fig: temporal-subcycling}
\end{figure}

  \begin{figure}[h]
    \centering
    \includegraphics[width=0.7\linewidth]{figures/png/subcycle_parareal_convergence.png}
    \caption{Estimated errors during parareal iterations for different amounts of subcycling.}
    \label{fig: temporal-subcycling-errors}
\end{figure}

\section{Parallel Scaling}
In chapter \ref{cha: pint}, two strategies are discussed to simulate long time frames; [A] each processor gets multiple coarse steps assigned to it, or [B] the time domain is split into separate chunks on which parareal is performed in serial. These can be used when the number of available cores is insufficient to assign a separate processor to each coarse time step. 
We will now discuss a core aspect of any parallel algorithm: the scalability with respect to the number of cores used. This information is crucial in deciding which strategy is most fitting for the desired simulation time period. This section assumes that the number of time steps equals the number of cores. Figure \ref{fig: core scaling} demonstrates how the speedup (left) and parallel efficiency (right) change depending on the number of time steps. The speedup steadily rises with the increase in cores, however, the parallel efficiency shows that this increase is not proportionate with the extra cores used. It insinuates that it can be more efficient to split up the time domain into multiple sequential parareal sections when long-duration simulations are desired. To remedy this, one could consider larger time step sizes for the coarse solver to more quickly cross large time spans. Special care should be taken, that the convergence of parareal is not impeded due to getting too close to the CFL condition. The decrease in parallel efficiency might necessitate the use of multiple computing nodes using \texttt{MPI} or offloading to GPU. This would constitute a subject for future work. Again, the theoretical bound is shown, where the parallel efficiency is approximated with $\frac{1}{k}$. Keeping this in mind, we can see that the number of iterations needed for larger time frames only increases once from $2$ to $3$, which is shown by the theoretical bound jumping from $0.5$ to $0.33$. This indicates that the reason the parallel efficiency is not constant, is not due to an increase in iterations. It is most likely caused by parallelisation overhead.
  \begin{figure}[h]
    \centering
    \includegraphics[width=1\linewidth]{figures/png/core_scaling_test.png}
    \caption{Speedup (left) and parallel efficiency (right) of the parareal algorithm for an increasing number of cores, where each core is assigned to one coarse time step.}
    \label{fig: core scaling}
\end{figure}

%%% ============================================================================================ %%%

\section{Linear Solvers}
\label{sec: linear solvers}
Here, we investigate the performance of some widely used (direct and indirect) solvers to approximate the solution to the linear system of equations obtained by discretising Maxwell's equations. 
% Direct solvers have the benefit of computing the solution \textit{exactly}; however, they may be computationally prohibitive for large systems. In a serial context, the solution should be accurate up to machine precision due to the desired exact energy conservation. Since the basic parareal algorithm does not conserve this property \cite{gander_analysis_2014}, it would have to iterate to highly stringent tolerances to achieve machine precision accuracy. Because total energy conservation is not a requirement for this work, more options are available to solve this linear system. The matrix resulting from the linear system is sparse and has a block-diagonal structure, which lends itself to iterative solvers, such as the generalised minimal residual method (GMRES). Unfortunately, the discretisation matrix is not symmetric, invalidating specific solvers, such as Cholesky-based decompositions. 
We consider the LU decomposition method along with the commonly used iterative solvers, GMRES and BiCGSTAB outlined in Chapter \ref{cha: methodology}. We investigate the effect of choosing different combinations of these three, for the coarse and fine solvers. It is known that the computational cost of an iterative solver depends on the time step size \cite{einkemmer_adaptive_2018, hochbruck_exponential_1998}, whereas the cost of a direct solver is independent of it. We also test different preconditioners to speed up the convergence of the iterative solvers.

\subsection{Combinations of Linear Solvers}
We first consider the use of different linear solvers. Specific combinations of direct and iterative solvers could allow for better speedup or computational runtime. A possible strategy is to use a very accurate but slow direct LU solver for the fine solution for high accuracy while using a less accurate iterative solver in the coarse. However, a coarse timestep could lead to more iterations needed for the iterative solvers to converge to the desired accuracy, thus increasing the overall computational time. Figure \ref{fig: linear solver combination} shows the results of the same simulation using each linear solver combination. The left panel shows the speedup achieved, and the right panel illustrates the computational runtime for each combination under consideration. Each cluster of three bars indicates the use of a different fine solver. Each blue bar indicates a simulation where the coarse solver uses an LU solver, the orange bars represent GMRES solvers, and the green corresponds to BiCGSTAB. While both bar plots are important, the right plot is essential in understanding the influence of the different solvers. The bar plot on the left clearly shows a preference for a costly LU solver for the fine integrator. However, this may be misleading as one has to consider the overall computational runtime. The figure on the right shows that it is best to use iterative solvers for both the fine and coarse solvers, and there is no significant difference between GMRES and BiCGSTAB. This discrepancy in interpretation is caused by the difference in computational runtime caused by the multiple fine solvers. 
As expected of parareal, the solutions are all equally accurate to their respective serial solutions. 
  \begin{figure}[h]
    \centering
    \includegraphics[width=1\linewidth]{figures/png/solver_test.png}
    \caption{Left: Speedup for parareal simulation using different combinations of linear solvers for coarse and fine. Right: Time needed for a parareal simulation using different combinations of linear solvers for coarse and fine.}
    \label{fig: linear solver combination}
\end{figure}

\subsection{Preconditioners}
We now investigate the influence of different preconditioners on the computational runtime of the linear solver.
Preconditioners are transformations performed on the linear system to improve the convergence rate of linear solvers. Here, we define preconditioners to be the inverse of a matrix $\textbf{P}$ defined so that $\textbf{P}^{-1}\textbf{A}$ has an increased convergence rate for iterative methods that solve $\textbf{Ax} = \textbf{b}$. This means a preconditioned system will solve $\textbf{P}^{-1}\textbf{Ax} = \textbf{P}^{-1}\textbf{b}$. Three different types of preconditioners will be tested. The first preconditioner is the identity matrix, i.e., no preconditioning. The second is the Jacobi preconditioner, for which the preconditioning matrix corresponds to the diagonal of the matrix $\textbf{A}$; it should perform well for diagonally dominant matrices. The final preconditioner that is examined is the incomplete LU factorisation. The general idea of incomplete LU is to factorise $\textbf{A}$ into the product of a lower and upper triangular matrix such that $\textbf{A} \approx \textbf{LU}$, where $\textbf{L}$ and $\textbf{U}$ share a sparsity pattern with $\textbf{A}$. This reduces memory and computational cost requirements whilst providing a rough estimate of the solution. One can then use this solution as a starting point.

Figure \ref{fig: preconditioners} shows the performance of the three preconditioners under consideration for different time step sizes for a serial implementation of CASE IV. While there is no significant difference between the identity and Jacobi preconditioners, one can see that the incomplete LU decomposition is faster when the time step size is larger and slower than the other preconditioners when the step size is reduced. While the incomplete LU preconditioner is more expensive to calculate, the iterative solver only needs a small number of iterations afterwards to solve the system during the test. This contrasts the identity and Jacobi preconditioners, which are fast to perform but might not help the solver converge fast enough, leading to many iterations when the initial guess is inaccurate.
 \begin{figure}[h]
    \centering
    \includegraphics[width=1\linewidth]{figures/png/precond_test.png}
    \caption{Time needed for serial solve of CASE IV using different preconditioners for GMRES.}
    \label{fig: preconditioners}
\end{figure}

% \section{Discussion}
% While the previous section showed the results of the different experiments, they will be linked together in this section. There are, after all, interesting interplays between the results that must not be overlooked. For a first example of the interconnection, one must look at the behaviour of Parareal for the non-smooth initial conditions outlined in \ref{sec: convergence}. It is clear that the convergence of Parareal strongly depends on the smoothness of the solution; in the case of non-smoothness, Parareal requires an accurate coarse solver, or it will not converge in an orderly fashion, sometimes even diverging. As a result one would prefer to use Parareal on smooth problems. If this is not the case, a more accurate coarse solver is needed, thus lowering the possible speedup. This, however, already determines the mode in which the linear solver will have to operate, more likely in the region where iterative solvers with quick preconditioners are best. The required small timesteps for the coarse solver also mean that many steps would be needed to cover more extended periods, which was found to not help the speedup of the algorithm. This means a more relaxed "CFL"-esque condition would significantly increase the number of possible implementations and use cases. \color{red} This this effect might be caused by particles moving so fast between time points that cells never experience their presence in the coarse solver. In this case, it might be possible to decrease the effect by increasing the b-splines' order in the particle-in-cell method so that a single particle influences more grid cells. Also, check whether the "CFL" condition is worse for parareal or the same as the linear.\color{black}


%%% Local Variables: 
%%% mode: latex
%%% TeX-master: "thesis"
%%% End:
\chapter{Conclusion}
\label{cha: conclusion}
This thesis concludes with the affirmation that the parareal algorithm is effectively applicable to the ECSIM for plasma simulations, yielding accurate results for a range of problems exploring plasma dynamics. While exact energy conservation remains a challenge, we establish that the error in energy can be tightly bound relative to the error in the state variables.

We assess the parallel performance of the parareal algorithm when used with ECSIM. Given that solving Maxwell's equations addresses hyperbolic equations, which typically exhibit poor convergence properties with parareal, our findings are particularly noteworthy. We demonstrate that significant speedup is possible with the optimal selection of coarse and fine integrators, which can be the same numerical simulation technique with varying time step sizes. Reducing the time step size of the fine solver enhances the speedup of parareal, though this benefit must be weighed against the computational cost for highly accurate solutions. Conversely, increasing the coarse solver's time step size is beneficial only up to a point beyond which additional parareal iterations negate the speedup gains to an extent.

Our investigation reveals that subcycling in the coarse solver generally decreases speedup due to the associated parallel overhead despite achieving increased accuracy. Optimal performance is achieved using iterative solvers for both coarse and fine integrators, coupled with inexpensive preconditioners, mainly due to the relatively small coarse time step size constrained by ECSIM's accuracy constraints. This small step size is necessary to ensure proper convergence for parareal. The coarse propagator must be sufficiently accurate to avoid accuracy constraints which inhibit the convergence of parareal. Under these conditions, we obtain results showing a parallel efficiency of $0.28$ on a simulation of 96 time steps using $\Delta t_\mathrm{Coarse} = 10^{-4}$, $\Delta t_\mathrm{Fine} \approx 2\cdot 10^{-7}$ and $N_x = 512$. This simulation only requires $3$ parareal iterations.

Future research could further enhance performance by employing methods with less restrictive accuracy constraints. This would allow for cheaper coarse solvers and validate alternative preconditioners and linear solvers. This could lead to even greater computational efficiencies and broader applicability of the parareal algorithm in plasma simulations. Other parallel-in-time methods which include coarsening in the spatial domain, e.g. PFASST, are also known to provide better performance results than parareal.
 Offloading to GPU could also improve performance as this opens up the possibility of using thousands of cores on a single node. However, we do notice diminishing returns for increasing core count.

%%% Local Variables: 
%%% mode: latex
%%% TeX-master: "thesis"
%%% End: 


% If you have appendices:
\appendixpage*          % if wanted
\appendix
% \chapter{Spatial coarsening}
\label{app: spatial coarsening}
Appendices hold useful data which is not essential to understand the work
done in the master's thesis. An example is a (program) source.
An appendix can also have sections as well as figures and references\cite{h2g2}.


%%% Local Variables: 
%%% mode: latex
%%% TeX-master: "thesis"
%%% End: 

% ... and so on until
\chapter{Energy-Conservation}
\label{app: regular grid}
Our derivation follows the formal proof of energy conservation of Lapenta (2017) \cite{lapenta_exactly_2017}. We decide to take some factors into account, which are ignored in the original, which allows us to make a more accurate comparison with the underlying Poynting's theorem.

We begin with the velocity update rule of the particle mover
\begin{equation}
    \textbf{v}_p^{n+1} = \textbf{v}_p^n + \frac{q_p \Delta t}{m_p}\left(\textbf{E}^{n+\theta}(\textbf{x}_p^{n+\frac{1}{2}}) + \bar{\textbf{v}}_p \times \textbf{B}^n(\textbf{x}_p^{n+\frac{1}{2}})\right)
\end{equation}
For energy conservation, we set $\theta = \frac{1}{2}$, which allows us to use the midpoint approximation $\textbf{E}^{n+\theta} = \bar{\textbf{E}} = \frac{\textbf{E}^{n+1} + \textbf{E}^{n}}{2}$. Next, we take the dot product of both sides with $\bar{\textbf{v}}_p = \frac{\textbf{v}^{n+1}_p + \textbf{v}_p^{n}}{2}$ and rearrange the terms.
\begin{equation}
    \frac{\|\textbf{v}_p^{n+1}\|^2 - \|\textbf{v}_p^{n}\|^2}{2 m_p \Delta t}= q_p \bar{\textbf{v}}_p \cdot \left(\bar{\textbf{E}}(\textbf{x}_p^{n+\frac{1}{2}}) + \bar{\textbf{v}}_p \times \textbf{B}^n(\textbf{x}_p^{n+\frac{1}{2}})\right)
\end{equation}
Using the properties of the cross product, $\bar{\textbf{v}}_p \cdot (\bar{\textbf{v}}_p \times \textbf{B}^n(\textbf{x}_p^{n+\frac{1}{2}})) = 0$, the equation simplifies to:
\begin{equation}
    \frac{\|\textbf{v}_p^{n+1}\|^2 - \|\textbf{v}_p^{n}\|^2}{2 m_p \Delta t}= q_p \bar{\textbf{v}}_p \cdot \bar{\textbf{E}}(\textbf{x}_p^{n+\frac{1}{2}})
\end{equation}
Summing over all particles and using the interpolation function to replace $\bar{\textbf{E}}(\textbf{x}_p^{n+\frac{1}{2}})$:
\begin{equation}
     \sum_p \frac{\|\textbf{v}_p^{n+1}\|^2 - \|\textbf{v}_p^{n}\|^2}{2 m_p \Delta t}= \sum_p q_p \bar{\textbf{v}}_p \cdot \sum_g \bar{\textbf{E}}_g W_{pg}
\end{equation}
We take the summation over the grid cells outside of the summation of the particles. Then, the definition of the current density is used to substitute the right hand side $\bar{\textbf{J}}_g = \frac{1}{V_g} \sum_p q_p \bar{\textbf{v}}_p W_{pg}$. We make sure to include the $\frac{1}{V_g}$ factor because the current density is used in the field solver instead of the current.
\begin{equation}
\label{eq: subs kinetic}
     \sum_p \frac{\|\textbf{v}_p^{n+1}\|^2 - \|\textbf{v}_p^{n}\|^2}{2 m_p \Delta t}= \sum_g \bar{\textbf{E}}_g \bar{\textbf{J}}_g V_g
\end{equation}
We now attempt to obtain a similar result starting from the field solver equations.
\begin{equation}
\left\{\begin{aligned} 
 	\nabla_g \times \bar{\mathbf{E}} +\frac{1}{c}\frac{\mathbf{B}_g^{n+1} - \mathbf{B}_g^{n}}{\Delta t} &= 0 \\
 	\nabla_g \times \bar{\mathbf{B}} -\frac{1}{c}\frac{\mathbf{E}_g^{n+1} - \mathbf{E}_g^{n}}{\Delta t} &= \frac{4 \pi}{c}\bar{\mathbf{J}}_g\\
 \end{aligned}\right.
\end{equation}
The first equation is dotted with $\bar{\textbf{B}}_g = \frac{\textbf{B}^{n+1}_g + \textbf{B}^{n}_g}{2}$ and the second with $\bar{\textbf{E}}_g = \frac{\textbf{E}^{n+1}_g + \textbf{E}^{n}_g}{2}$. The rearranged result can be written as
\begin{equation}
\left\{\begin{aligned} 
 	\frac{\|\mathbf{B}_g^{n+1}\|^2 - \|\mathbf{B}_g^{n}\|^2}{2 c \Delta t} &= -\bar{\textbf{B}}_g \cdot \nabla_g \times \bar{\mathbf{E}} \\
 	\frac{\|\mathbf{E}_g^{n+1}\|^2 - \|\mathbf{E}_g^{n}\|^2}{2 c \Delta t} &= -\frac{4 \pi}{c}\bar{\textbf{E}}_g \cdot \bar{\mathbf{J}}_g + \bar{\textbf{E}}_g \cdot \nabla_g \times \bar{\mathbf{B}}  \\
 \end{aligned}\right.
\end{equation}
Summing the two equations and rearranging the factors gives
\begin{align}
    &\frac{\|\mathbf{B}_g^{n+1}\|^2 - \|\mathbf{B}_g^{n}\|^2}{8 \pi \Delta t} + \frac{\|\mathbf{E}_g^{n+1}\|^2 - \|\mathbf{E}_g^{n}\|^2}{8 \pi \Delta t} \notag \\
     &= -\bar{\textbf{E}}_g \cdot \bar{\mathbf{J}}_g - \frac{c}{4\pi} \left( \bar{\textbf{B}}_g \cdot \nabla_g \times \bar{\mathbf{E}}- \bar{\textbf{E}}_g \cdot \nabla_g \times \mathbf{B}^{n+ \theta} \right)
\end{align}
We now use the property $\nabla_g \cdot (\bar{\textbf{E}} \times \bar{\mathbf{B}}) = \bar{\textbf{B}}_g \cdot (\nabla_g \times \bar{\mathbf{E}})- \bar{\textbf{E}}_g \cdot (\nabla_g \times \bar{\mathbf{B}})$
\begin{equation}
        \frac{\|\mathbf{B}_g^{n+1}\|^2 - \|\mathbf{B}_g^{n}\|^2}{8 \pi \Delta t} + \frac{\|\mathbf{E}_g^{n+1}\|^2 - \|\mathbf{E}_g^{n}\|^2}{8 \pi \Delta t} = -\bar{\textbf{E}}_g \cdot \bar{\mathbf{J}}_g - \frac{c}{4\pi} \nabla_g \cdot (\bar{\textbf{E}} \times \bar{\mathbf{B}}) 
\end{equation}
We then multiply everything by $V_g$ and sum over all cells
\begin{align}
        &-\sum_g \frac{\left(\|\mathbf{B}_g^{n+1}\|^2 + \|\mathbf{E}_g^{n+1}\|^2 - \|\mathbf{E}_g^{n}\|^2 - \|\mathbf{B}_g^{n}\|^2\right) V_g}{8 \pi \Delta t} \notag \\
         &=\sum_g \bar{\textbf{E}}_g \cdot \bar{\mathbf{J}}_g V_g + \frac{c}{4\pi} \sum_g V_g \nabla_g \cdot (\bar{\textbf{E}} \times \bar{\mathbf{B}}) 
\end{align}
This is the discretised Poynting's theorem, showing that the change in electromagnetic energy in the system is equal to the work performed on the charges plus the flow of energy out of the system. However, from equation \ref{eq: subs kinetic} we know what the work is that is performed on the charges. 
\begin{align}
	&-\sum_g \frac{\left(\|\mathbf{B}_g^{n+1}\|^2 + \|\mathbf{E}_g^{n+1}\|^2 - \|\mathbf{E}_g^{n}\|^2 - \|\mathbf{B}_g^{n}\|^2\right) V_g}{8 \pi \Delta t} - \sum_p \frac{\|\textbf{v}_p^{n+1}\|^2 - \|\textbf{v}_p^{n}\|^2}{2 m_p \Delta t} \notag \\
	&= \frac{c}{4\pi} \sum_g V_g \nabla_g \cdot (\bar{\textbf{E}} \times \bar{\mathbf{B}})
\end{align}
We perform one final substitution where we introduce the total energy contained in the discretised system at a time point $t_n$ as $P^n = \sum_p \frac{\|\textbf{v}_p^{n}\|^2}{2 m_p} + \sum_g \frac{\|\mathbf{E}_g^{n}\|^2 V_g}{8 \pi} + \sum_g \frac{\|\mathbf{B}_g^{n}\|^2 V_g}{8 \pi}$. The terms represent, in order, the kinetic, electric and magnetic energy contained in the system (using Gaussian units). 
\begin{equation}
        -\frac{P^{n+1} - P^{n}}{\Delta t}= \frac{c}{4\pi} \sum_g V_g \nabla_g \cdot (\bar{\textbf{E}} \times \bar{\mathbf{B}}) 
\end{equation}
With this notation and interpretation, we can see that the negative, discretised derivative of the total energy at a time point $t_{n + \theta}$ is equal to the discretisation of the energy flow out of the system. This indicates that if the energy flow out of the system is 0, then the discretised ECSIM algorithm will be energy conserving as well.

%%% Local Variables: 
%%% mode: latex
%%% TeX-master: "thesis"
%%% End: 


\backmatter
% The bibliography comes after the appendices.
% You can replace the standard "abbrv" bibliography style by another one.
\bibliographystyle{abbrv}
\bibliography{references}

\end{document}

%%% Local Variables: 
%%% mode: latex
%%% TeX-master: t
%%% End: 
