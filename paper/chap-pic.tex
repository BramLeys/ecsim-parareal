\chapter{Particle-In-Cell simulations of Plasma}
\label{cha: pic}

\color{red}
Particle-in-cell, PIC, methods are closely related to the kinetic approach and rely on the Vlasov equation as well.
PIC methods combine many particles close to each other in phase space into superparticles. These superparticles have a finite shape, and their interaction with others weakens as their overlap increases. The shape of a superparticle, $p$,is determined in phase space by the \textit{shape functions} for \textbf{x} and \textbf{v}, $S_{\textbf{x}}(\textbf{x}-\textbf{x}_p(t))$ and $S_{\textbf{v}}(\textbf{v}-\textbf{v}_p(t))$ respectively. It is assumed that the actual physical distribution can be represented as the superposition of these superparticles:
\[f(\textbf{x}(t),\textbf{v}(t),t) = \sum_{p}N_pS_{\textbf{x}}(\textbf{x}-\textbf{x}_p(t))S_{\textbf{v}}(\textbf{v}-\textbf{v}_p(t))\]
The evolution of these particles follows the laws of motion\eqref{eq:cont particle mover} while conserving the total number of particles represented by the superparticle.
\begin{equation} \label{eq:cont particle mover}
\begin{split}
	\diff{{\textbf{x}_p}}{t} &= \textbf{v}_p \\
	\diff{{\textbf{v}_p}}{t} &= \frac{1}{m_p}\left(\textbf{F} + q_p(\textbf{E} + \textbf{v}\times \textbf{B})\right)
\end{split}
\end{equation}
These equations describe the movement of the particles across the domain, and as such the calculation of the positions and velocities of the particles is referred to as the "particle mover" part. The section corresponding to the computation of the self-consistent electric and magnetic fields is called the "field solver". The electric field, \textbf{E}, and magnetic field, \textbf{B}, are calculated by solving the Maxwell equations:
\begin{align}
	\nabla \cdot \mathbf{E} &= \frac{\rho}{\varepsilon_0} \\
	\nabla \cdot \mathbf{B} &= 0 \\
	\nabla \times \mathbf{E} &= -\frac{1}{c}\frac{\partial \mathbf{B}}{\partial t} \\
	\nabla \times \mathbf{B} &= \frac{1}{c}\left(\mathbf{J} + \frac{\partial \mathbf{E}}{\partial t}\right)
\end{align}
\color{blue}write something about the fact that you use normalized units\color{black} 
\color{red}
Calculating these in practice requires discretization. The domain is discretised in space using a regular grid. On this grid, the electric field is stored at the vertices, while the magnetic field is kept at the centres. Particles can then move across this grid, and the force experienced by these particles is computed by interpolating the electric and magnetic fields at the neighbouring grid points. The standard \textit{interpolation function} from a particle, $p$, to a grid point , $g$, is defined as\cite{lapenta_exactly_2017} \[W(\textbf{x}_g - \textbf{x}_p) = \int S_{\textbf{x}}(\textbf{x}-\textbf{x}_p)b_0\left(\frac{\textbf{x}-\textbf{x}_g}{\Delta\textbf{x}}\right) d\textbf{x}\],
where $b_0$ is the b-spline of order 0. This means that if the shape function is a b-spline of order $l$, $S_\textbf{x} = \frac{1}{\Delta x}b_l\left(\frac{\textbf{x}-\textbf{x}_p}{\Delta x}\right))$, meaning the particle shape function is the same size as the grid cell size, then the interpolation function can be simplified using the properties of b-splines to 
\[W(\textbf{x}_g - \textbf{x}_p) =b_{l+1}\left(\frac{\textbf{x}-\textbf{x}_g}{\Delta \textbf{x}}\right)\]

The integration of the differential equations sets apart all of the different PIC methods. For example, a simple algorithm could use the \textit{leap-frog scheme} for the particle mover: 
\begin{align*}
	\textbf{x}_p^{n+1} &= \textbf{x}_p^n + \Delta t \textbf{v}_p^{n+\frac{1}{2}} \\
	\textbf{v}_p^{n+\frac{3}{2}} &= \textbf{v}_p^{n+\frac{1}{2}} + \Delta t \frac{q_p}{m_p}\left(\frac{\textbf{F}(\textbf{x}_p^{n+1})}{q_p} +\textbf{E}_p(\textbf{x}_p^{n+1}) + \bar{\textbf{v}}_p \times \textbf{B}(\textbf{x}_p^{n+1})\right)\\
\end{align*},
where $\bar{\textbf{v}}_p=\frac{\textbf{v}_p^{n+\frac{3}{2}}+\textbf{v}_p^{n+\frac{1}{2}}}{2}$.
The position and velocity are calculated at different time steps. This staggering makes the method second-order accurate since it is essentially a centred difference. The electric and magnetic fields can be found by solving the following equations \cite{jiang_origin_1996}:
  \begin{align*}
	\nabla_g \times \mathbf{E}^{n} +\frac{1}{c}\frac{\mathbf{B}^{n+1} - \mathbf{B}^{n}}{\Delta t} &= 0 \\
	\nabla_g \times \mathbf{B}^{n} -\frac{1}{c}\frac{\mathbf{E}^{n+1} - \mathbf{E}^{n}}{\Delta t} &= \frac{4 \pi}{c}\bar{\mathbf{J}}_g\\
\end{align*}
Depending on how the current at each grid point, $\bar{\mathbf{J}}_g$, is calculated also heavily influences the properties of the method. For example, energy conservation heavily depends on accurately capturing the non-linear interaction between particles and fields. If energy is not conserved, then extra care needs to be taken so that results obtained by this method would not be too different from reality. While multiple PIC methods are energy-conserving, most are not direct and require some form of linear or non-linear iterations. \cite{giovanni_lapenta_introduction_nodate}

\subsection{Energy Conserving Semi-Implicit Method}
\label{subsec: plasma intro ECSIM}
The energy-conserving semi-implicit method, ECSIM, is fully energy-conserving, unconditionally stable, and only requires a linear solver \cite{lapenta_exactly_2017}. It is based on iPIC3D \cite{markidis_multi-scale_2010} and the energy conserving PIC $\theta$-scheme \cite{brackbill_implicit_1982}. Its particle mover is given by 
\begin{align*}
	\textbf{x}_p^{n+\frac{1}{2}} &= \textbf{x}_p^{n-\frac{1}{2}} + \Delta t \textbf{v}_p^{n} \\
	\textbf{v}_p^{n+1} &= \textbf{v}_p^{n} + \Delta t \frac{q_p}{m_p}\left(\textbf{E}^{n+\theta}_p(\textbf{x}_p^{n+\frac{1}{2}}) + \bar{\textbf{v}}_p \times \textbf{B}^n(\textbf{x}_p^{n+\frac{1}{2}})\right)\\
\end{align*}
and its field solver is: 
   \begin{align*}
 	\nabla_g \times \mathbf{E}^{n + \theta} +\frac{1}{c}\frac{\mathbf{B}^{n+1} - \mathbf{B}^{n}}{\Delta t} &= 0 \\
 	\nabla_g \times \mathbf{B}^{n+ \theta} -\frac{1}{c}\frac{\mathbf{E}^{n+1} - \mathbf{E}^{n}}{\Delta t} &= \frac{4 \pi}{c}\bar{\mathbf{J}}_g\\
 \end{align*}
 While other algorithms keep some non-linearities or linearize the current to ensure energy conservation, ECSIM calculates the current without approximation. It does this by using a \textit{mass matrices}, $M_{gg'}$, the elements of which are given by (using the shorthand notation $W(\textbf{x}_g- \textbf{x}_p)= W_{gp}$): \[
 M^{ij}_{g g'} = \sum_p q_p \alpha_p^{ij,n} W_{g'p} W_{gp}\]
 The elements $\alpha_p^{ij,n}$ are part of the rotation matrix that allows for $\bar{\textbf{v}}_p$ to be written as \[\bar{\textbf{v}}_p = \hat{\textbf{v}}_p + \frac{q_p \Delta t}{2 m_p}\hat{\textbf{E}}_p\]
 where 
 \[ \hat{\textbf{v}}_p = \alpha^n_p \textbf{v}_p^n, \quad \hat{\textbf{E}}_p = \alpha^n_p \textbf{E}_p^{n+\theta}
 \]
 Using these mass matrices, the current can be calculated as 
 \[\bar{\textbf{J}}_g = \frac{1}{V_g}\left(\sum_p q_p \hat{\textbf{v}}_p W_{pg} + \frac{q_p \Delta t}{2 m_p}\sum_{g'}M_{g g'} \textbf{E}^{n+\theta}_{g'}\right)\]
The ECSIM algorithm is also compatible with smoothing operations and subcycling. Subcycling is the act of not performing the field equations in every time step. This cuts down on the computational cost in exchange for accuracy. Since particles normally follow a gyrating path in the presence of the electromagnetic fields, the subcycling can also be used to step over the gyration cycle for the fields, while being able to use an average of the position during the gyration while calculating the fields in the next time step.
\color{black}


%%% Local Variables: 
%%% mode: latex
%%% TeX-master: "thesis"
%%% End: 
